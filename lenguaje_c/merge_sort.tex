% Copyright (C) 2010-2013, Maximiliano Curia <maxy@gnuservers.com.ar>,
%               2010-2013, Margarita Manterola <marga@marga.com.ar>

% Esta obra está licenciada de forma dual, bajo las licencias Creative
% Commons:
%  * Atribución-Compartir Obras Derivadas Igual 2.5 Argentina
%    http://creativecommons.org/licenses/by-sa/2.5/ar/
%  * Atribución-Compartir Obras Derivadas Igual 3.0 Unported
%    http://creativecommons.org/licenses/by-sa/3.0/deed.es_AR.
%
% A su criterio, puede utilizar una u otra licencia, o las dos.
% Para ver una copia de las licencias, puede visitar los sitios
% mencionados, o enviar una carta a Creative Commons,
% 171 Second Street, Suite 300, San Francisco, California, 94105, USA.

\renewcommand{\chaptermark}[1]{\markboth{#1}{}}
\renewcommand{\thesection}{\arabic{section}}
\chapter*{Ordenamiento recursivo por mezcla, \textit{merge sort}}

El método de ordenamiento \textit{merge sort} es uno de los métodos de
ordenamiento recursivos, basados en la técnica de dividir y conquistar.  Se
lo puede utilizar para ordenar cualquier estructura secuencial (vectores,
listas, etc).

Los pasos de este método de ordenamiento son:
\begin{enumerate}
\item Cuando la longitud del vector sea 0 o 1, se considera que ya se
encuentra ordenado. De no ser así:
\item Se divide el vector en dos partes de aproximadamente la mitad del
tamaño.
\item Se ordena cada una de esas partes, utilizando este mismo método.
\item Tomando las dos partes ordenadas, se las intercala de forma ordenada,
para obtener el vector original ordenado.
\end{enumerate}

Por ejemplo, si el vector original es \verb+[6, 7, -1, 0, 5, 2, 3, 8]+, se
lo dividirá en dos partes: \verb+[6, 7, -1, 0]+ y \verb+[5, 2, 3, 8]+, que
serán ordenadas de forma recursiva.  Luego de ordenarlas, se obtendrá:
\verb+[-1, 0, 6, 7]+ y \verb+[2, 3, 5, 8]+.  Al intercalar ordenadamente
los dos vectores ordenados se obtendrá la solución buscada:
\verb+[-1, 0, 2, 3, 5, 6, 7, 8]+.

\section{Implementación básica}

Será necesario programar dos funciones.  Por un lado la función
\lstinline!merge_sort!, que será la función recursiva encargada de dividir
la lista en dos hasta llegar a la condición de corte (cuando la lista tenga
un tamaño menor que 2).

\begin{codigo-c}
void merge_sort(int vector[], int inicio, int fin)
{
    int largo = fin - inicio;
    if (largo < 2) {
        return;
    }

    int medio = inicio + (largo / 2);
    merge_sort (vector, inicio, medio);
    merge_sort (vector, medio, fin);
    merge (vector, inicio, medio, fin);
}
\end{codigo-c}

Se puede ver que esta es una función extremadamente sencilla, cuya única
tarea es dividir el vector en dos partes. Por otro lado, será necesario
programar la función \lstinline!merge!, que será la encargada de intercalar
las partes una vez que estén ordenadas.

\begin{codigo-c}
void merge (int vector[], int inicio, int medio, int fin)
{
    int pos_1 = inicio;
    int pos_2 = medio;
    int aux[fin - inicio];
    int pos_a = 0;

    // Intercala ordenadamente
    while ( (pos_1 < medio) && (pos_2 < fin) ) {
        if ( vector[pos_1] <= vector[pos_2] ) {
            aux[pos_a] = vector[pos_1];
            pos_a++; pos_1++;
        } else {
            aux[pos_a] = vector[pos_2];
            pos_a++; pos_2++;
        }
    }
    // Copia lo que haya quedado al final del primer vector
    while (pos_1 < medio) {
        aux[pos_a] = vector[pos_1];
        pos_a++; pos_1++;
    }
    // Copia lo que haya quedado al final del segundo vector
    while (pos_2 < fin) {
        aux[pos_a] = vector[pos_2];
        pos_a++; pos_2++;
    }

    // Copia los valores del vector auxiliar al original
    int a = 0;
    int i = inicio;
    while (i < fin) {
        vector[i] = aux[a];
        i++; a++;
    }
}
\end{codigo-c}

Si bien esta función tiene unas cuantas líneas de código, su tarea no es
muy compleja, simplemente inserta en un vector auxiliar los elementos del
vector que ya se encuentran ordenados, de forma que sólo se recorren los
elementos una sola vez.

\section{Análisis de complejidad}

Sea $N$ la longitud del vector. Se pueden hacer las siguientes
observaciones:

\begin{itemize}
\item En la función \lstinline!merge! se ve que el tiempo que se tarda en
intercalar dos vectores de longitud $N/2$ es proporcional a $N$, ya que
todos los elementos se copian una vez al vector auxiliar y luego se los
vuelve a copiar al vector original. Es posible, entonces, utilizar $A * N$
para representar ese tiempo.

\item Si se denomina $T(N)$ al tiempo que tarda el algoritmo en ordenar
un vector de longitud $N$, en la función \lstinline!merge_sort!
se puede ver que $T(N) = 2 * T(N/2) + A * N$, ya que la función simplemente
se llama a sí misma con dos partes de la mitad de tamaño, y luego a la
función \lstinline!merge! con el tamaño total.

\item Además, también en \lstinline!merge_sort! se puede ver que el tiempo
necesario para un vector de longitud menor a 2 es sólo el necesario en
hacer una comparación. Es decir que, $T(1) = T(0) = B$.
\end{itemize}

Estos datos forman una ecuación de recurrencia, para resolverla, se
supondrá que $N = 2^k$, quedando las ecuaciones:

\begin{eqnarray}
T(N) = T(2^k) &=&  2 * T(2^{k-1}) + A * 2^k \\
T(1) &=& B
\end{eqnarray}

Es posible resolver estas ecuaciones utilizando el \textit{método
telescópico}.

\begin{eqnarray}
T(2^k) &=& 2 * T(2^{k-1}) + A * 2^k \\
&=& 2*(2*T(2^{k-2} )+A*2^{k-1} )+A*2^k\\
&=& 2^2*T(2^{k-2} )+ 2*A(2^k)\\
&\vdots&\\
&=& 2^i* T(2^{k-i})+ i * A * 2^i\\
&\vdots&\\
&=&2^k*T(1) + k * A * 2^k\\
&=&2^k*B  + k * A * 2^k
\end{eqnarray}

Como $N = 2^k$ entonces $k=\log_2N$, y por lo que esta resolución demuestra
que $T(N) = B*N+A*N*\log_2N$.

Como $A*N*\log_2N$ es un término de mayor orden que $B*N$, el orden de este
algoritmo es $O(N*\log_2N)$.

Los valores de las constantes $A$ y $B$ son importantes a la hora de buscar
la mejor implementación de \textit{merge sort}, pero no para el cálculo del
orden del algoritmo.

Por otro lado, al analizar el espacio que consume este algoritmo, se puede
ver que para realizar el intercalado, necesita copiar el vector a un
vectora auxiliar, es decir que duplica el espacio consumido.

\section{Implementaciones más eficientes}

Si bien el orden del algoritmo \textit{merge sort} será siempre
$O(N*\log_2N)$, si se quiere una implementación realmente eficiente del
ordenamiento, será necesario hacerle algunas mejoras a la implementación
mostrada.

\subsection{Implementación con un solo pedido de memoria}

El valor $A$ está asociado al tiempo necesario para ejecutar la función
\lstinline!merge!. Una de las operaciones que se puede eliminar de esta
función es el pedido de memoria para el arreglo auxiliar,
\lstinline!int aux[fin-inicio];!, ya que esta operación consume tiempo en
pedir la memoria (y luego devolverla, al terminar la función), que podría
ahorrarse si se hiciera un único pedido de memoria para todo el algoritmo.

Para ello, será necesario crear una función adicional, que sea la que haga
el pedido de memoria auxiliar, y luego llame a la función recursiva, con
esa memoria ya reservada.

\begin{codigo-c}
void merge_sort(int vector[], int largo)
{
    int aux[largo];
    msort(vector, 0, largo, aux);
}
\end{codigo-c}

La función recursiva es ahora \lstinline!msort!, que es prácticamente igual
a la vista previamente, siemplemente incluye el pasaje de la variable
auxiliar.

\begin{codigo-c}
void msort(int vector[], int inicio, int fin, int aux[])
{
    int largo = fin - inicio;
    if (largo == 1) {
        aux[inicio] = vector[inicio];
        return;
    }

    int medio = inicio + (largo / 2);
    msort (vector, inicio, medio, aux);
    msort (vector, medio, fin, aux);
    merge (vector, inicio, medio, fin, aux);
}
\end{codigo-c}

Por otro lado, la función \lstinline!merge!, ya no deberá realizar el
pedido de memoria para alojar el vector adicional, sino que trabajará
directamente sobre la misma porción del vector auxiliar que la utilizada
para el vector de valores.

\begin{codigo-c}
void merge (int vector[], int inicio, int medio, int fin, int aux[])
{
    int pos_1 = inicio;
    int pos_2 = medio;
    int pos_a = inicio;

    // Intercala ordenadamente (...)
    // Copia lo que haya quedado al final del primer vector (...)
    // Copia lo que haya quedado al final del segundo vector (...)

    // Copia los valores del vector auxiliar al original
    int i;
    for (i = inicio; i < fin; i++) {
        vector[i] = aux[i];
    }
}
\end{codigo-c}

De esta forma se logró evitar tener que estar pidiendo memoria para el
vector auxiliar una y otra vez.  Sin embargo, esto no alcanza para decir
que se cuenta con una versión realmente eficiente de \textit{merge sort}.

\subsection{Otras mejoras}

Se puede seguir trabajando sobre el mismo algoritmo para agregarle varias
otras mejoras, como por ejemplo:

\begin{description}

\item[Implementación sin copia inútil de los datos]

Otra operación que consume tiempo inútilmente es volver a copiar los
datos del vector auxiliar al principal al terminar la función
\lstinline!merge!.

Esta copia puede evitarse si se opera alternadamente
con el vector auxiliar y con el principal, de modo que el vector auxiliar
de una llamada a \lstinline!msort! es el principal de la llamada recursiva
realizada dentro de la función, y así.

\item[Uso de otros tipos de datos]

En los ejemplos mostrados se han usado vectores de enteros para hacer más
simple el ejemplo, pero de la misma forma puede usarse cualquier otro tipo
de dato que tengamos alguna forma de compararlo. O bien hacer una
implementación que no le importe el tipo de dato con el que opera, y use
una función que recibe por parámetro para comparar elementos.

\end{description}

