% Copyright (C) 2010-2013, Maximiliano Curia <maxy@gnuservers.com.ar>,
%               2010-2013, Margarita Manterola <marga@marga.com.ar>

% Esta obra está licenciada de forma dual, bajo las licencias Creative
% Commons:
%  * Atribución-Compartir Obras Derivadas Igual 2.5 Argentina
%    http://creativecommons.org/licenses/by-sa/2.5/ar/
%  * Atribución-Compartir Obras Derivadas Igual 3.0 Unported
%    http://creativecommons.org/licenses/by-sa/3.0/deed.es_AR.
%
% A su criterio, puede utilizar una u otra licencia, o las dos.
% Para ver una copia de las licencias, puede visitar los sitios
% mencionados, o enviar una carta a Creative Commons,
% 171 Second Street, Suite 300, San Francisco, California, 94105, USA.

\renewcommand{\chaptermark}[1]{\markboth{#1}{}}
\renewcommand{\thesection}{\arabic{section}}
\chapter*{Compilación de varios archivos en C}

En todo programa es importante modularizar el código de forma que se facilite
la reutilización y se minimice la repetición de código.
En particular, cuando se trata de tipos abstractos de datos, es importante
tener un módulo correspondiente a cada tipo abstracto.

Por otro lado, para que un tipo abstracto de datos sea realmente
\textit{abstracto} es recomendable que la implementación de las operaciones
correspondientes al tipo estén separadas de los prototipos de estas
operaciones, de modo que quien las utiliza se concentre únicamente en cuáles
son las operaciones y no en cómo se llevan a cabo.

\section{Encabezado, implementación y código objeto}

En C, esto se logra mediante la separación de cada módulo en un archivo
\verb!.h!, que contiene las declaraciones de estructuras, constantes y
prototipos de las funciones, que es llamado el \textit{encabezado} del módulo,
y un archivo \verb!.c! que contiene las implementaciones correspondientes.

Cada uno de los archivos \verb!.c! se utiliza para generar un archivo
\verb!.o! que contiene el \textit{código objeto}, es decir el código de
máquina, correspondiente a cada módulo.

Cuando se compila un programa completo, todos los \verb!.o! que se hayan
generado a partir de los módulos programados deben combinarse en un sólo
ejecutable.

\subsection{Inclusión de otros encabezados}

Cuando un módulo requiere de funciones definidas en otros módulos, debe
incluir los encabezados (los archivos \verb!.h!) en los que esas funciones
están definidas.  Esta inclusión se realiza normalmente dentro del encabezado
correspondiente al módulo en cuestión.

Es posible que al momento de construir un programa de tamaño considerable
suceda que hay varios módulos que dependen de otro. De modo que podría suceder
que este otro se incluya varias veces, lo cual es problemático y debe ser
evitado.

Para solucionar este problema, se utilizan las construcciones condicionales
del preprocesador, haciendo que el código definido dentro de un encabezado se
incluya en el programa un única vez. Por ejemplo:

\begin{codigo-c-plano}
#ifndef __ENUM_H
    #define __ENUM_H
    typedef enum {OK, ERROR} estado;
    typedef enum {HUMANO, COMPUTADORA} jugador_t;
#endif
\end{codigo-c-plano}

En este caso, el preprocesador verifica si está definida la variable
\lstinline!__ENUM_H!. De no estar definida, la define y luego define los tipos
enumerados correspondientes a este encabezado.

En cambio, si la variable ya estaba definida, significa que este encabezado ya
fue procesado, con lo cual no se hace nada.


\section{Compilación con \texttt{make}}

Cuando los módulos que componen un programa son muchos, los pasos necesarios
para compilarlo pueden ser muchos y tener que regenerarlos manualmente cada
vez que se los modificque sería una tarea demasiado tediosa.  Es por ello que
existe una herramienta llamada \verb!make!, encargada de realizar todos los
pasos de compilación y necesarios y de hacerlos sólo cuando haga falta.

Esta herramienta utiliza, a modo de configuración, los archivos
\verb!Makefile! en donde se indican los pasos a realizar para la compilación
tanto de los módulos como del programa principal.

En estos archivos, básicamente, se pueden definir variables y reglas para
compilar los distintos módulos.

\subsection{Un \texttt{Makefile} sencillo}

A continuación un ejemplo de cómo puede verse un posible archivo
\verb!Makefile!.

\begin{lstlisting}[language=make, numbers=none]
CFLAGS = -g -Wall -std=c99
EXEC = miprog
OBJ = lista.o pila.o
CC = gcc

all: $(EXEC)

lista.o: lista.c lista.h
	$(CC) $(CFLAGS) -c lista.c

pila.o: pila.c pila.h
	$(CC) $(CFLAGS) -c pila.c

$(EXEC): $(OBJ) miprog.c
	$(CC) $(CFLAGS) $(OBJ) miprog.c -o $(EXEC)
\end{lstlisting}

\subsection{Variables}

En la primera parte se declaran 4 variables, \lstinline!CFLAGS! son los
\textit{flags} (parámetros) de compilación utilizados, en este caso se trata
del parámetro que incluye la información para depuración \lstinline!-g! y el
parámetro para que advierta sobre todos los posibles problemas que el
compilador encuentre, \lstinline!-Wall!.

Luego se declara el nombre que tendrá el programa ejecutable.  En este caso no
tiene extensión, puesto que es un programa para sistemas UNIX (Linux, BSD,
Solaris, etc). Si se estuviera compilando para un sistema Windows, la variable
\lstinline!EXEC! sería \lstinline!miprog.exe!.

Luego se listan cuáles serán los módulos que deberán transformarse en código
objeto, y finalmente se coloca el nombre del compilador.  Separar la
información de esta manera permite que si es necesario hacer un cambio en la
forma de compilar un programa, el trabajo para realizarlo sea mínimo.

Es importante notar que en el \verb!Makefile!, las variables se definen
simplemente con \lstinline!VARIABLE=VALOR!, pero luego para utilizarlas, se lo
hace de la forma \lstinline!$(VARIABLE)!.

\subsection{Reglas de compilación}

Luego de las variables, se definen cada uno de los archivos a generar, los
archivos de los cuales estos dependen, y las acciones a llevar a cabo para
compilar los archivos correspondientes.

Cuando se ejecuta el comando \verb!make! sin ningún parámetro, se ejecuta
automáticamente la primera de todas las reglas, es por eso que esta regla
normalmente se llama \lstinline!all! y simplemente indica cuál es el archivo a
generar.

Luego de esta regla especial, se encuentran las reglas de compilación de cada
uno de los módulos del programa.  Las reglas tienen un formato específico que
se debe cumplir para que funcione el \verb!Makefile!.

\begin{lstlisting}[language=make, numbers=none]
archivo: dependencias
	acciones
\end{lstlisting}

Esto significa que \lstinline!archivo! debe generarse cada vez que
cambie una de las \lstinline!dependencias!, ejecutando las
\lstinline!acciones!.

Es importante notar que para que el \verb!make! funcione correctamente, las
acciones a realizar deben tener un tabulador de separación desde el comienzo
de línea.  Puede haber más de una acción, de a una por línea, siempre que se
mantenga un tabulador de separación. \\

En particular, la regla que se encarga de generar el programa principal es
diferente a las otras, ya que incluye una mayor cantidad de dependencias y de
variables.

\begin{lstlisting}[language=make, numbers=none]
$(EXEC): $(OBJ) miprog.c
	$(CC) $(CFLAGS) $(OBJ) miprog.c -o $(EXEC)
\end{lstlisting}

Esta regla indica que el archivo indicado mediante la variable
\lstinline!$(EXEC)! definida previamente debe generarse cuando cambie cualquiera
de los archivos objeto, ya que si estos cambian, también debe cambiar el
ejecutable final, o bien si cambia el código principal del programa. \\

Es importante notar que mediante estas reglas, \verb!make! no sólo es capaz de
compilar los archivos de forma correcta, sino que también es capaz de
realizarlo sólo cuando sea necesario.  Para ello, verifica que la fecha de
modificación de los archivos listados como dependencias sea anterior al
archivo a generar.  De no ser así, vuelve a generarlo, ya que algo ha
cambiado.

\subsection{Reglas genéricas}

Cuando la gran mayoría de los archivos del programa se compilan de una misma
forma, es posible simplificar el archivo \verb!Makefile!, mediante el uso de reglas
genéricas.

Por ejemplo, para el caso del \verb!Makefile! visto anteriormente, las dos
líneas que generan los archivos \verb!.o! podrían simplificarse en una sola de
la siguiente forma:

\begin{lstlisting}[language=make, numbers=none]
%.o: %.c %.h
	$(CC) $(CFLAGS) -c $<
\end{lstlisting}

Esta regla significa que para generar un archivo \verb!.o! es necesario contar
con un archivo del mismo nombre \verb!.c! y otro del mismo nombre \verb!.h!.
En la regla de compilación se utiliza la variable especial \lstinline!$<!, que
toma el valor de la primera de las dependencias listadas (el archivo
\verb!.c!.).  Existe también \lstinline!$@!, que toma el nombre del archivo
que está siendo generado en esa regla.

\subsection{Acciones comunes}

Además de compilar, es común querer eliminar los archivos generados, de modo
que queden únicamente los archivos fuente del programa.  Esta acción
normalmente se realiza mediante una regla especial llamada \lstinline!clean!.

Si bien puede llevar cualquier nombre, lo más usual es ponerle este nombre ya
que tanta gente la llama así que cualquier programador que se encuentre con un
\verb!Makefile! esperará encontrar una regla con este nombre que realice esta
acción.

\begin{lstlisting}[language=make, numbers=none]
clean:
	rm $(OBJ) $(EXEC)
\end{lstlisting}

Para utilizar esta regla (o cualquier otra que no sea la predeterminada), debe
invocarse el comando \verb!make! con el nombre de la regla como parámetro. Es
decir, \verb!make clean!.

Al igual que con \lstinline!clean!, existen otras acciones comunes que se
suelen incluir en la mayoría de los programas.  Las más conocidas:

\begin{description}
\item[build] Para compilar el código (equivalente a la regla
\lstinline!all! del ejemplo mostrado).
\item[install] Para instalar el código compilado en el sistema.
\item[uninstall] Para desinstalar el programa que haya sido
instalado.
\end{description}

\subsection{Variables comunes}

Así como existen reglas comunes, que la mayoría de los programadores están
acostumbrados a encontrar en los archivos \verb!Makefile!, también existen
variables que suelen estar presentes.

\begin{description}
\item[CFLAGS] Que estaba en el ejemplo mostrado, son los
parámetros pasados al compilador.
\item[LDFLAGS] Son los parámetros pasados al enlazador.
\item[PREFIX] Utilizado cuando hay una regla de instalación, es el
directorio a partir del cual se instalarán los archivos.
\end{description}

\subsection{Reglas, archivos y PHONY}

Cada regla de makefile tiene como finalidad crear un archivo, sin embargo hay
reglas como la regla \verb!clean! que mostramos más arriba que no genera
ningún archivo, tampoco depende de ningún archivo (de hecho borra archivos,
pero esto es parte de la acción y a make no le importa que hace la acción).

Es más, si creamos un archivo llamado \verb!clean! make creerá que la regla
está satisfecha. Para evitar que \verb!make! revise archivos que no vamos a
generar es de bastante recomendable declarar las reglas que no generan
archivos como \verb!.PHONY!, por ejemplo:

\begin{lstlisting}[language=make, numbers=none]
.PHONY: install clean
\end{lstlisting}

