% Copyright (C) 2010-2013, Maximiliano Curia <maxy@gnuservers.com.ar>,
%               2010-2013, Margarita Manterola <marga@marga.com.ar>

% Esta obra está licenciada de forma dual, bajo las licencias Creative
% Commons:
%  * Atribución-Compartir Obras Derivadas Igual 2.5 Argentina
%    http://creativecommons.org/licenses/by-sa/2.5/ar/
%  * Atribución-Compartir Obras Derivadas Igual 3.0 Unported
%    http://creativecommons.org/licenses/by-sa/3.0/deed.es_AR.
%
% A su criterio, puede utilizar una u otra licencia, o las dos.
% Para ver una copia de las licencias, puede visitar los sitios
% mencionados, o enviar una carta a Creative Commons,
% 171 Second Street, Suite 300, San Francisco, California, 94105, USA.

\renewcommand{\chaptermark}[1]{\markboth{#1}{}}
\renewcommand{\thesection}{\arabic{section}}
\chapter*{Ordenamiento rápido, \textit{quick sort}}

El método de ordenamiento \textit{quick sort} es el más famoso de los
métodos de ordenamiento recursivos, su fama se basa en que puede ser
implementado de forma muy eficiente y en la gran mayoría de los casos tiene
el mismo orden de complejidad que \textit{merge sort}.
Al igual que este último, está basado en la técnica de dividir y conquistar.\\

Los pasos de este método de ordenamiento son:
\begin{enumerate}
\item Cuando la longitud del vector sea 0 o 1, se considera que ya se
encuentra ordenado. De no ser así:
\item Se elige un elemento del vector como \textit{pivote}.  Generalmente
será el primero o el último.
\item Se reordenan los elementos del vector de modo que quede dividido en
tres partes (\textbf{partición}): los elementos menores al pivote, el pivote y los elementos
mayores al pivote. Al terminar este paso, el pivote
queda en su lugar definitivo.
\item Se repite el mismo proceso para cada una de las partes que no
contienen al pivote (los menores y los mayores).
\end{enumerate}

Por ejemplo, si el vector original es \verb+[6, 7, -1, 0, 5, 2, 3, 8]+ y se
elige el primer elemento como pivote (\verb!6!), la partición del vector
será: \verb![-1, 0, 5, 2, 3]!, \verb!6!, \verb![7, 8]!. Se procederá a
ordenar recursivamente \verb![-1, 0, 5, 2, 3]! y \verb![7, 8]!, de modo que
el vector final será \verb![-1, 0, 2, 3, 5, 6, 7, 8]!.

\section{Implementación básica}

En este caso se implementará una función \lstinline!quick_sort!, que se
encargará tanto de realizar la partición, como de llamarse recursivamente
hasta que no haya más elementos que ordenar.

La elección del pivote depende de cada implementación de \textit{quick
sort}, en este caso se elige el primer elemento del vector como pivote.

\begin{codigo-c}
void quick_sort(int vector[], int inicio, int fin)
{
    int pivote = inicio;
    int ult_menor = inicio;

    if ( (fin - inicio) < 2 ) {
        return;
    }

    int i;
    for (i = pivote + 1; i < fin; i++) {
        if ( vector[i] < vector[pivote] ) {
            ult_menor++;
            swap(vector, i, ult_menor);
        }
    }

    // Coloca el pivote al final de los menores y el último
    // menor en el primer lugar.
    swap(vector, pivote, ult_menor);
    // Ordena cada una de las mitades
    quick_sort(vector, inicio, ult_menor);
    quick_sort(vector, ult_menor+1, fin);

}
\end{codigo-c}

El bucle principal de la función recorre los elementos del vector una única
vez, cambiando de lugar aquellos que son menores al pivote para que queden
en la primera parte y que los mayores queden en la segunda.

Una vez terminado este bucle, se coloca el pivote en el medio de ambas
partes, de modo que quede ubicado en su posición final.

La función \lstinline!swap! utilizada en esta porción de código, recibe un
vector y dos posiciones dentro del vector, e intercambia los valores que se
encuentran en esas dos posiciones:

\begin{codigo-c}
void swap(int vector[], int pos_1, int pos_2)
{
    int aux = vector[pos_1];
    vector[pos_1] = vector[pos_2];
    vector[pos_2] = aux;
}
\end{codigo-c}

Esta función puede utilizarse siempre que se necesite intercambiar dos
elementos de un vector.

\section{Análisis de complejidad}

A simple vista, el algoritmo de \textit{quick sort} puede parecer muy
similar al de \textit{merge sort}, ya que en ambos casos se divide a la
lista en dos, y se opera sobre partes cada vez más pequeñas.

Sin embargo, algo importante a tener en cuenta es que en el caso del
\textit{quick sort} el orden que tenga el algoritmo dependerá en una gran
parte de la elección del pivote, ya que no es lo mismo elegir un valor que
se encuentre aproximadamente en el medio, de forma que las dos partes sean
aproximadamente del mismo tamaño, que elegir un valor que se encuentre en
uno de los extremos, de modo que una de las partes mida mucho más que la
otra.

Asumiendo que el valor elegido se encuentra aproximadamente en el medio,
se puede ver que el tiempo requerido para ejecutar el algoritmo es:

\begin{eqnarray}
T(N) &=& A * N + 2*T(N/2) \\
T(1) &=& B
\end{eqnarray}

Donde $B$ es el tiempo requerido por el caso base, y $A$ es el valor asociado
a recorrer el vector y cambiar los elementos de lugar en el bucle
principal.  Puede verse que estas ecuaciones son las mismas que para
\textit{merge sort}. \\

Sin embargo, cuando el pivote elegido no divide ambas partes al medio, el
comportamiento no es tan bueno.  En el peor caso (cuando una parte queda
con todos los elementos menos el pivote y la otra vacía), será:

\begin{eqnarray}
T(N) &=& A * N + T(N-1) \\
T(1) &=& B
\end{eqnarray}

De modo que aplicando el método telescópico, similar al utilizado
anteriormente:

\begin{eqnarray}
T(N) &=& A * N + T(N-1) \\
&=& A * N + A * (N-1) + T(N-2) \\
&=& A * (N + N - 1 + N - 2) + T(N-3) \\
&\vdots&\\
&=&A * (N + N - 1 + N - 2 + \hdots + 2) + B\\
&=&A * \sum_{i=2}^N{i} + B\\
&=&A * \frac{N^2+N}{2} - 1 + B
\end{eqnarray}

Se puede ver que en el peor caso, el orden será $O(N^2)$, mucho peor que el
$O(Nlog_2N)$ visto anteriormente.  Sin embargo, se pueden tomar recaudos
especiales para que este peor caso sea extremadamente improbable, y que en
la práctica se pueda considerar que el algoritmo se comporta como
$O(Nlog_2N)$.

\section{Implementaciones más eficientes}

\subsection{Elección del pivote}

Si se elige el primer elemento (o el último), el algoritmo resulta muy
inconveniente para el caso de una lista que ya se encuentra ordenada, y
este es un caso que en ciertas situaciones es esperable que suceda.

Es por eso que una optimización sencilla es intercambiar el elemento del
medio con el que se vaya a utilizar de pivote antes de comenzar el bucle
principal.

\begin{codigo-c}
    int medio = (fin + inicio) / 2;
    swap(vector, pivote, medio);
\end{codigo-c}

Otras técnicas de elección del pivote incluyen:
\begin{itemize}
\item Elegir un elemento aleatoriamente, esto hace que en promedio sea
mucho más probable tener un buen caso que uno malo, pero no elimina la
posibilidad del peor caso.
\item Recorrer la lista y buscar el elemento que ocupará la posición
central de la lista.  Eso asegura que el orden sea siempre $O(Nlog_2N)$, pero
decrementa mucho la eficiencia del caso base.
\item Elegir tres elementos de la lista (por ejemplo, el primero, el del
medio y el último), y quedarse con el del medio de los tres como pivote.
\end{itemize}

\subsection{Reducción de la cantidad de intercambios}

En la implementación vista, puede suceder que se hagan numerosos
intercambios innecesarios, cuando un elemento ya es menor que el pivote, y
simplemente haría falta avanzar la variable que indica la posición del
último menor.

Una implementación alternativa de {\it quick sort} se basa en esta idea para
tratar de minimizar la cantidad de intercambios.  Se cuenta con dos
variables, que se utilizan para saltear los elementos que no hace falta
cambiar de lugar, y sólo cambiar aquellos que es necesario.

\begin{codigo-c}
void quick_sort(int vector[], int inicio, int fin)
{
    if ( (fin - inicio) < 2 ) {
        return;
    }
    int izq = inicio + 1;
    int der = fin - 1;
    int pivote = inicio;

    // Cambia el del medio con el primero.
    // (optimización para vectores ordenados).
    int medio = (izq + der) / 2;
    swap(vector, pivote, medio);

    while (izq <= der) {
        while ( (izq <= der) && (vector[der] >= vector[pivote]) )
            der--;
        while ( (izq <= der) && (vector[izq] < vector[pivote]) )
            izq++;
        if ( izq < der )
            swap(vector, izq, der);
    }

    // Coloca el pivote al final de los menores y el último
    // menor en el primer lugar.
    swap(vector, pivote, der);
    // Ordena cada una de las mitades
    quick_sort(vector, inicio, der);
    quick_sort(vector, der+1, fin);
}
\end{codigo-c}

Como se puede ver, se ha reemplazado el bucle principal, por otro que
recorre el vector desde ambas puntas hacia el medio, buscando los elementos
que necesitan ser intercambiados.

\subsection{Utilización de otros algoritmos}

En particular para los casos de las partes más pequeñas, al tener menos
elementos, sin importar cuál se elija como pivote, es más probable que se
asemejen al peor caso.

Es por ello que una técnica de optimización puede incluir utilizar un
algoritmo alternativo, como por ejemplo el de ordenamiento por inserción,
para secuencias de pocos elementos. \\

Por otro lado, puede también implementarse un contador que verifique la
profundidad de la recursión y cuando esta exceda el nivel esperado por el
algoritmo, pasar a utilizar otro algoritmo de ordenamiento, como {\it merge
sort} o {\it heap sort}.

\section{Quick sort en la biblioteca estándar de C}

Entre las funciones que provee la biblioteca estándar de C, se incluye una
implementación de quick sort. Dado que la biblioteca estándar está altamente
probada y seguramente contenga optimizaciones avanzadas, es en general una
buena idea usar las funciones que provistas antes que usar las propias.

En el caso del \textit{quick sort}, la función se llama \lstinline!qsort! y se
encuentra definida en el encabezado \lstinline!<stdlib.h>!, su prototipo es:

\begin{codigo-c}
void qsort(void *base, size_t nmemb, size_t size,
           int(*compar)(const void *, const void *));
\end{codigo-c}

Que puede ser intimidante por la cantidad y complejidad de parámetros que
recibe, en gran parte debido a la generalidad del código. \lstinline!base! se
refiere al vector a ordenar, \lstinline!nmemb! es la cantidad de elementos
del vector, \lstinline!size! es el tamaño en bytes de un elemento,
\lstinline!compar! es la función que se debe usar para comparar.

Este tipo de funciones se verán con más detalle más adelante.

