% Copyright (C) 2010-2013, Maximiliano Curia <maxy@gnuservers.com.ar>,
%               2010-2013, Margarita Manterola <marga@marga.com.ar>

% Esta obra está licenciada de forma dual, bajo las licencias Creative
% Commons:
%  * Atribución-Compartir Obras Derivadas Igual 2.5 Argentina
%    http://creativecommons.org/licenses/by-sa/2.5/ar/
%  * Atribución-Compartir Obras Derivadas Igual 3.0 Unported
%    http://creativecommons.org/licenses/by-sa/3.0/deed.es_AR.
%
% A su criterio, puede utilizar una u otra licencia, o las dos.
% Para ver una copia de las licencias, puede visitar los sitios
% mencionados, o enviar una carta a Creative Commons,
% 171 Second Street, Suite 300, San Francisco, California, 94105, USA.

\renewcommand{\chaptermark}[1]{\markboth{#1}{}}
\renewcommand{\thesection}{\arabic{section}}
\chapter*{Manejo de memoria en C}

Todas las variables, en el lenguaje C, se definen dentro de alguna función,
fuera de esa función no es posible acceder a ellas.  Al entrar a una función,
a cada una de las variables definidas en esa función se le asigna el espacio
que sea necesario dentro de una pila interna de memoria (\textit{stack}) con
la que cuenta el programa, y al terminar la función se desapila todo lo
definido en ella. Es decir que la pila crece con cada llamado a una función y
decrece con cada función que se termina.

Por otro lado, todos los tipos que C define como parte del lenguaje son de un
tamaño fijo, incluso los definidos por el usuario usando \lstinline!struct!.
Es por eso que el espacio que se reserva en la pila interna de memoria tiene
un tamaño fijo.

Existe, además, otro espacio de memoria que se utiliza cuando el tamaño de los
datos no es fijo.  A este espacio de memoria dinámica se lo llama
\textit{heap}, contiene bloques de memoria, que el programador puede solicitar
para utilizar según sea conveniente.

\section{Obtener el tamaño de un tipo (\textit{sizeof()})}

El operador \lstinline!sizeof()! devuelve el tamaño en bytes de un tipo de
datos (como \lstinline!int!). Por comodidad se le puede pasar tanto el nombre
de una variable o el nombre de un tipo de datos, en ambos casos devolverá el
tamaño del tipo de datos asociado.

\begin{codigo-c-plano}
 int largo, a;
 largo = sizeof(int);  // Cantidad en bytes de int
 largo = sizeof(a);    // Identico a lo anterior
\end{codigo-c-plano}

En este caso, ambas llamadas a \lstinline!sizeof! devuelven el tamaño que
ocupa un entero en memoria en la arquitectura y compilador que se esté
utilizando (por lo general son 4 bytes).

\begin{codigo-c-plano}
 char c;
 largo = sizeof(c); // Cantidad de bytes de un char
\end{codigo-c-plano}

En este caso, se devuelve cuánto ocupa un caracter.  Los caracteres ocupan
siempre 1 byte.

\begin{codigo-c-plano}
 char *puntero;
 largo = sizeof(puntero); // Cantidad de bytes usados por un puntero.
\end{codigo-c-plano}

En este caso, se devuelve la cantidad de bytes que ocupa un puntero. Todos los
punteros tienen el mismo tamaño, y es tal que pueda contener cualquier
dirección de memoria, sea estática o dinámica.

\begin{codigo-c-plano}
 int vector[100];
 largo = sizeof(vector);                // Cantidad de bytes del vector
 largo = sizeof(vector) / sizeof(int);  // Largo del vector (100)
\end{codigo-c-plano}

En este caso, la primera llamada devuelve el tamaño total en bytes ocupado por
el vector (usualmente serían 400 bytes), mientras que la segunda devuelve
siempre 100 sin importar la plataforma, ya que divide el espacio total del
vector por el tamaño de cada uno de los elementos.

\begin{atencion}
Los vectores no siempre se comportan como punteros. En el ejemplo anterior
\lstinline!vector! no se refiere a la posición de memoria donde comienza el
vector, sino a todo el vector. Este es el comportamiento esperado para los
vectores estáticos definidos en el mismo entorno, esto quiere decir, que si el
vector lo recibieramos en una función y le hacemos \lstinline!sizeof(vector)!
el resultado sería la cantidad de bytes usados por un puntero.
\end{atencion}

\section{Memoria dinámica en C}

Mediante la utilización de los punteros, es posible  acceder a cualquier
porción de memoria válida, tanto si se encuentra dentro de la pila interna
como si se encuentra dentro del espacio de memoria dinámica.

Para obtener una porción de memoria válida dentro del espacio de memoria
dinámica, existen en la biblioteca estándar funciones (\lstinline!malloc! y
\lstinline!realloc!) que reservan un bloque de memoria y devuelven su
dirección. Utilizando estas funciones es posible conseguir estructuras de
datos dinámicas, que pueden variar su tamaño según sea necesario, en lugar de
tener un tamaño ya definido.

Es importante notar que la memoria dinámica reservada mediante las funciones
de la biblioteca estándar no es liberada automáticamente, quien haya hecho la
reserva de memoria debe encargarse también de liberarla (con la
correspondiente función de biblioteca estándar, \lstinline!free!), de no ser
así, se dice que el programa \textit{pierde memoria}, ya que los bloques
reservados no pueden volverse a utilizar aún cuando ya no estén en uso.

\subsection{Pedir memoria al sistema (\textit{malloc()})}

Para pedir memoria al sistema se utiliza la función
\lstinline!malloc!\footnote{Para más información: \texttt{man 3 malloc}.},
definida en \lstinline!<stdlib.h>!, cuyo prototipo es el siguiente:

\begin{codigo-c-plano}
void *malloc(size_t tamanio);
\end{codigo-c-plano}

Si el sistema tiene suficiente memoria disponible, \lstinline!malloc! devuelve
un puntero a la primera posición de memoria de un bloque de memoria dinámica,
de \lstinline!tamanio! bytes.

Si ocurriera algún problema, porque el sistema no dispone de suficiente memoria
o similar, la llamada de \lstinline!malloc! devolvería \lstinline!NULL!.

\section{Devolver memoria al sistema (\textit{free()})}

Cuando un bloque de memoria ya no es necesario para un programa, se lo debe
devolver al sistema, de forma que el sistema pueda tenerlo nuevamente entre los
recursos a utilizar por otros procesos. La función
\lstinline!free!\footnote{Para más información: \texttt{man 3 free}.} de la
biblioteca estándar hace exactamente esto, su prototipo es:

\begin{codigo-c-plano}
void free(void *puntero);
\end{codigo-c-plano}

Recibe como único parámetro un puntero antes devuelto por \lstinline!malloc! o
\lstinline!realloc!, libera ese bloque de memoria y no devuelve nada. \\

Es importante notar que no es posible liberar una porción de memoria que ya ha
sido liberada.  Esta acción puede provocar el mismo tipo de errores que los
provocados por acceder a una porción de memoria inválida. 

\section{Agrandar/achicar un bloque de memoria (\textit{realloc()})}

Cuando se necesita modificar el tamaño de un bloque memoria, se utiliza la
función \lstinline!realloc!\footnote{Para más información: \texttt{man 3
realloc}.} de la biblioteca estándar. Su prototipo es el siguiente:

\begin{codigo-c-plano}
void *realloc(void *puntero_anterior, size_t nuevo_tamanio);
\end{codigo-c-plano}

Recibe un puntero antes obtenido mediante \lstinline!malloc! o
\lstinline!realloc! y el nuevo tamaño del bloque de memoria. Si todo
funciona bien, devuelve un nuevo puntero al nuevo bloque de memoria, copia el
contenido del bloque viejo al nuevo (copia el largo mínimo entre los dos
bloques), y el bloque anterior es liberado.

Si algo falla, devuelve \lstinline!NULL!, y el bloque anterior no se modifica.

Teniendo en cuenta este comportamiento, normalmente \textbf{no} se utiliza una
construcción como la siguiente.

\begin{codigo-c-plano}
/* MAL */
datos = (dato_t*) realloc( datos, sizeof(dato_t) * tamanio_nuevo);
\end{codigo-c-plano}

Ya que si \lstinline!realloc! devolviera \lstinline!NULL!, existirá en memoria
un bloque de datos válido, con información válida, al cual es imposible
acceder, ya que se perdió la dirección de memoria que antes estaba en
\lstinline!datos!.

En cambio se debe utilizar:

\begin{codigo-c-plano}
void *aux = realloc( datos, sizeof(dato_t) * tamanio_nuevo );
if ( aux == NULL ) {
	// realloc no pudo pedir el bloque nuevo, hacer algo al respecto.
} else {
	datos = (dato_t*) aux;
}
\end{codigo-c-plano}

Otro posible problema a tener en cuenta es cuando se tienen punteros que
apuntan a \textbf{partes} de un bloque de memoria y se llama a
\lstinline!realloc! sobre ese bloque: los punteros pasarán a ser inválidos, ya
que apuntan a direcciones que ya no son las del bloque en cuestión.

De modo que hay que tener mucho cuidado de nunca guardar referencias a
porciones de memoria que pueden ser movidas de lugar mediante
\lstinline!realloc!.

\section{Ejemplo: una pila de tamaño variable}

Tener una pila de tamaño variable es un ejemplo sencillo de manejo de memoria
dinámica. La pila contiene un arreglo de valores, donde se van apilando y
desapilando los elementos; el problema se da cuando se quieren apilar más
elementos que los que la pila puede almacenar, en ese caso se debe reservar un
bloque de memoria de mayor tamaño para que siga siendo posible agregar
elementos a la pila.

En este caso, la estructura de la pila será de la forma:

\begin{codigo-c-plano}
typedef struct {
	int cantidad;
	dato_t *datos;
	int tamanio;
} pila_dinamica;
\end{codigo-c-plano}

Donde \lstinline!cantidad! es la cantidad de elementos almacenada en la pila,
mientras que \lstinline!tamanio! es el tamaño actual de la pila, es decir, la
máxima cantidad de elementos que puede almacenar antes de tener que reservar
una porción mayor de memoria.

\subsection{Creación de la pila}

Cuando se trabaja con memoria dinámica, las funciones de creación de una
estructura, no sólo deben inicializar los atributos de la estructura, sino que
también deben hacer el primer pedido de memoria, para reservar el bloque
inicial con el que se trabajará.

\begin{codigo-c-plano}
bool pila_crear(pila_dinamica *pila) 
{
	pila->cantidad = 0;
	pila->datos = (dato_t *) malloc( TAM_INICIAL * sizeof(dato_t) );
	if (pila->datos == NULL) return false;
	pila->tamanio = TAM_INICIAL;
	return true;
}
\end{codigo-c-plano}

La función \lstinline!malloc! es la encargada de reservar un bloque de memoria
para los datos que contendrá la pila.  El tamaño de este bloque de memoria es
el argumento que se le da a \lstinline!malloc!.  En este caso es el resultado
de multiplicar una constante \lstinline!TAM_INICIAL! por el tamaño del tipo de
dato que contiene la pila, es decir que en primera instancia la pila podrá
contener \lstinline!TAM_INICIAL! elementos.

% %%% Ya está explicado arriba, no hace falta volver a explicarlo %%%
%La función \lstinline!malloc! devuelve un puntero del tipo \lstinline!void *!,
%es decir un puntero genérico.  Es por eso que es necesario convertir ese
%puntero al puntero de tipo específico \lstiline!dato_t *!, para indicar que se
%trata de un puntero a datos de tipo \lstinline!dato_t!.

Si por algún motivo el sistema operativo no pudiera reservar la memoria
requerida, la función \lstinline!malloc! devuelve \lstinline!NULL!. En ese
caso, la función de creación de la pila devuelve \lstinline!false! para
indicar que no se ha podido crear la pila.

\subsection{Incremento del tamaño de la pila}

En el caso de agotarse el lugar, será necesario reservar una porción mayor de
memoria.  Es decir que la función \lstinline!apilar! deberá ser de la forma:

\begin{codigo-c-plano}
bool pila_apilar(pila_dinamica *pila, dato_t valor)
{
    if (pila->cantidad >= pila->tamanio) {
        if (! pila_cambiar_tamanio(pila, pila->tamanio*2)) return false;
    }
    pila->datos[pila->cantidad++] = valor;
    return true;
}
\end{codigo-c-plano}

En esta función, cuando la cantidad de elementos es igual o mayor al tamaño
actual de la pila, se llama a la función \lstinline!pila_cambiar_tamanio!, que
será la encargada de reservar un bloque de mayor tamaño, en este caso se le
pide que el bloque sea del doble del tamaño original. La función
\lstinline!pila_cambiar_tamanio! tendrá la siguiente forma:

\begin{codigo-c-plano}
bool pila_cambiar_tamanio(pila_dinamica *pila, size_t tamanio)
{
    void *aux = realloc (pila->datos, sizeof(dato_t) * tamanio);
    if (aux == NULL) return false;
    pila->datos = (dato_t*) aux;
    pila->tamanio = tamanio;
    return true;
}
\end{codigo-c-plano}

En este caso se utiliza la función \lstinline!realloc!, que recibe la
dirección actual donde se encuentran los datos, y el nuevo tamaño que se
quiere reservar.  \lstinline!realloc! se encarga de reservar el nuevo bloque,
copiar toda la información que estaba en el bloque viejo al nuevo y liberar el
viejo.

Al igual que en el caso de la creación de la pila, si por algún motivo no es
posible reservar la memoria según se quiere, \lstinline!realloc! devuelve
\lstinline!NULL!.  En este caso, los valores que ya estaban en la pila siguen
estando ahí, simplemente significa que no se ha podido agrandar la porción de
memoria reservada según se había pedido, es por ello que se utiliza un puntero
auxiliar \lstinline!aux! y sólo se lo asigna al atributo \lstinline!datos! en
el caso en que la reserva de memoria haya sido exitosa.

\subsection{Destrucción de la pila}

Como ya se dijo, cuando se reserva memoria mediante estas funciones, es
importante luego liberar la memoria reservada, porque de no hacerlo quedan
bloques de memoria inutilizables.  Es por ello que será necesario contar con
una función \lstinline!pila_destruir!, y quien utilice la pila deberá recordar
llamar a esta función al terminar de utilizarla.

\begin{codigo-c-plano}
bool pila_destruir(pila_dinamica *pila)
{
    free(pila->datos);
}
\end{codigo-c-plano}

Una vez que se ha llamado a la función \lstinline!free!, la porción de memoria
dinámica deja de estar reservada, ya no es más una porción de memoria válida y
no puede ser accedida por el programa (a menos que se haga una nueva reserva).  

\subsection{Disminución del tamaño de la pila}

Finalmente, si bien es posible tener una pila que sólo crezca y nunca se
reduzca, en general es deseable liberar la memoria que no está siendo
utilizada, para que pueda ser usada por otras partes del programa.  De modo
que sería deseable que la pila se reduzca al desapilar, cuando el espacio
ocupado por los elementos en mucho menor que el tamaño de la pila.

\begin{codigo-c-plano}
bool pila_desapilar(pila_dinamica *pila, dato_t *valor)
{
    if ( ! pila_ver_tope(pila, valor) ) return false;
    pila->cantidad--;
    if ( pila->cantidad < (pila->tamanio / 4) ) {
        pila_cambiar_tamanio(pila, pila->tamanio/2);
    }
    return true;
}
\end{codigo-c-plano}

En este caso, luego de desapilar el elemento pedido, la función verifica si la
cantidad de elementos ocupa menos de un cuarto del tamaño total de la pila, y
de ser así, reduce el tamaño a la mitad.

\section{Aritmética de punteros}

Las direcciones de memoria en C son valores enteros positivos, el valor
\lstinline!NULL! es equivalente a \lstinline!0!, que es una posición de memoria
inválida. 

Los punteros contienen direcciones de memoria asociadas a un tipo en
particular (a excepción de el tipo \lstinline!void!). Como ya se vio, cada
tipo tiene asociado un tamaño en bytes. El compilador de C utiliza esta
información para poder realizar operaciones aritméticas (sumas y restas de
valores enteros) con punteros.

Este es un tema que se presenta en principio complejo, pero que hace que se
pueda operar de forma muy poderosa sobre las porciones de memoria utilizadas.
En particular, es posible utilizar esta técnica para recorrer un arreglo de
valores, sin que necesariamente se los haya declarado como vector. Ejemplo:

\begin{codigo-c-plano}
float fs[MAX_LARGO];       
float *pf = fs;           // pf apunta al comienzo del vector fs
pf = pf + 1;              // pf apunta al segundo elemento de fs (&fs[1])
pf = fs + MAX_LARGO - 1;  // pf apunta a el último elemento de fs
\end{codigo-c-plano}

En este caso se declara un vector de valores de tipo \lstinline!float! (que
ocupan 4 bytes cada uno), luego se declara un puntero a valores de tipo
\lstinline!float!, que se inicializa con la posición de memoria del primer
elemento del vector.

Al sumarle 1, sin embargo, la posición de memoria se incrementa 4 bytes, ya
que se trata de un puntero y de esta manera avanza al siguiente
\lstinline!float! del vector.

De la misma manera, en la última línea, se obtiene un puntero a la dirección
de memoria del último elemento del arreglo. \\

Si no se declara el puntero del tipo correcto, en cambio, no es posible operar
de esta manera con las direcciones de memoria.  Es decir que si se hiciera
algo como lo siguiente:

\begin{codigo-c-plano}
void *pv = fs;  // pv apunta al comienzo del vector fs (sin tipo asociado)
pv = pv+1;      // pv apunta al segundo byte de fs *ERROR*
\end{codigo-c-plano}

Se obtendría un puntero a la dirección de memoria del segundo \textbf{byte}
del vector de \lstinline!float!, lo cual podría dar lugar a diversos errores,
ya que si se quiere acceder a la información, se estaría accediendo a 3 bytes
de un valor y 1 byte del siguiente. \\

Es por ello que es posible decir que el operador de acceso a un elemento
\lstinline![]! en C es \textit{azúcar sintáctico}\footnote{Un agregado a la
sintaxis del lenguaje para hacerla más agradable, pero no imprescindible.},
siendo \lstinline!a[10]! sintácticamente equivalente a \lstinline!*(a+10)!,
así como a \lstinline!10[a]!. Claro que este último, si bien válido, hace que
el código sea extremadamente poco legible, por lo que no se lo debe utilizar.

\section{Uso directo de bloques de memoria}

En la biblioteca estándar de C hay varias funciones útiles que acceden
directamente a la memoria, que permiten copiar o inicializar valores, que,
por lo general, serán necesarias cuando se trabaje con bloques de memoria.

Estas funciones asumen que tanto el bloque de memoria origen y destino son
porciones válidas de memoria, y que pueden ser escritas desde el programa.  De
no ser así, el sistema probablemente termine la ejecución del programa al
encontrar un acceso a una porción de memoria inválida, generalmente mediante
el error \textit{Violación de segmento} (\textit{Segmentation fault}).

Además, es importante tener en cuenta que todas estas operaciones tienen un
costo lineal con respecto al tamaño de memoria sobre el cual operan, es decir
que el tiempo requerido para ejecutarlas depende del tamaño de los bloques de
memoria.

\subsection{Copiar contenidos de bloques de memoria (\textit{memcpy()} y
\textit{memmove()})}

Para copiar el contenido de un bloque memoria a otro se puede utilizar la
función \lstinline!memcpy!\footnote{Para más información: \texttt{man 3
memcpy}.}, declarada en el encabezado \lstinline!<string.h>!, cuyo prototipo
es:

\begin{codigo-c-plano}
void *memcpy(void *destino, const void *origen, size_t cantidad);
\end{codigo-c-plano}

La función copia \lstinline!cantidad! bytes desde la posición de memoria
\lstinline!origen! hacia la posición \lstinline!destino! y devuelve un puntero
a \lstinline!destino!. La función asume que origen y destino son bloques que
no se solapan.

Cuando origen y destino se solapan se debe utilizar la función
\lstinline!memmove!\footnote{Para más información: \texttt{man 3 memmove}.},
también declarada en el encabezado \lstinline!<string.h>!, cuyo prototipo es:

\begin{codigo-c-plano}
void *memmove(void *destino, const void *origen, size_t cantidad);
\end{codigo-c-plano}

La función copia \lstinline!cantidad! bytes desde la posición de memoria
\lstinline!origen! hacia la posición \lstinline!destino! y devuelve un puntero
a \lstinline!destino!. De haber solapamiento, se encarga de que no se pierdan
los datos al momento de hacer la copia.

\subsection{Inicialización de un bloque de memoria (\textit{memset()})}

Para inicializar un bloque de memoria con un valor se puede utilizar la
función \lstinline!memset!\footnote{Para más información: \texttt{man 3
memset}.}, definida en el encabezado \lstinline!<string.h>!, cuyo prototipo
es:

\begin{codigo-c-plano}
void *memset(void *direccion, int byte, size_t cantidad);
\end{codigo-c-plano}

Que escribe el valor de \lstinline!byte!, en el bloque de 
\lstinline!cantidad! bytes, que empieza en \lstinline!direccion!.  Devuelve un
puntero a \lstinline!direccion!.

