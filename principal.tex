% Copyright (C) 2010-2013, Maximiliano Curia <maxy@gnuservers.com.ar>,
%               2010-2013, Margarita Manterola <marga@marga.com.ar>

% Esta obra está licenciada de forma dual, bajo las licencias Creative
% Commons:
%  * Atribución-Compartir Obras Derivadas Igual 2.5 Argentina
%    http://creativecommons.org/licenses/by-sa/2.5/ar/
%  * Atribución-Compartir Obras Derivadas Igual 3.0 Unported
%    http://creativecommons.org/licenses/by-sa/3.0/deed.es_AR.
%
% A su criterio, puede utilizar una u otra licencia, o las dos.
% Para ver una copia de las licencias, puede visitar los sitios
% mencionados, o enviar una carta a Creative Commons,
% 171 Second Street, Suite 300, San Francisco, California, 94105, USA.

\documentclass[11pt,spanish,a4paper]{report}
\tolerance=5000
%\renewcommand{\baselinestretch}{1.3}

% Entrada de texto
\usepackage[utf8]{inputenc}   % Permite escribir directamente áéíóúñ
\usepackage[spanish]{babel}   % Traduce los textos a castellano
\usepackage{t1enc}            % Agrega caracteres extendidos al font

% Cuestiones de estilo
\usepackage{palatino}         % Cambia el font por omision a Palatino
\usepackage{graphicx}         % Permite insertar imagenes.
\usepackage{listingsutf8}     % Permite mostrar codigo de forma mas linda
\usepackage{verbatim}         % Permite incluir archivos de texto verbatim

% Opcionales
\usepackage{longtable}         % tablas largas y flexibles
\usepackage{nonfloat}          % Hace que las figuras no floten
%\usepackage{fancyvrb}         % Agrega más entornos verbatim
%\usepackage{stmaryrd}         % Agrega más símbolos matemáticos
%\usepackage{enumerate}        % Agrega flexibilidad a las enumeraciones
%\usepackage{ifthen}           % Permite usar condicionales en comandos
%\usepackage{subfigure}        % Permite subfiguras dentro de las figuras

% Paquetes matematicos.  Hacen falta?
\usepackage{amsmath, amsthm, amssymb} % Se usan para theoremstyle

% Margenes y largo del texto
%
% medidas horizontales
% 1in(fijo) + \hoffset + \(odd|even)sidemargin + \textwidth + \marginparsep +
% \marginparwidth + \marginparpush
%
% medidas verticales
% 1in(fijo) + \voffset + \topmargin + \headheight + \headsep + \textheight +
% \footskip

\setlength\oddsidemargin{-0.04cm}
\setlength\evensidemargin{-0.04cm}
\setlength\topmargin{0cm}
\setlength\headheight{0.5cm}
\setlength\headsep{0.3cm}
\setlength\footskip{0.8cm}
\setlength\textwidth{16cm}  % ancho para apunte
\setlength\textheight{23cm} % largo para apunte
%\leftmargin 2.5cm
%\rightmargin 2.5cm

\usepackage{fancyhdr}         %
\pagestyle{fancy}
\fancyhf{} % clear all header and footer fields
\fancyhead[RO]{\slshape \rightmark \hspace{1cm} \normalfont \bfseries \thepage}
\fancyhead[LE]{\bfseries \thepage \hspace{1cm} \normalfont \leftmark}
%\fancyfoot{} % clear all footer fields
%\fancyfoot[RO]{\bfseries \thepage}
%\fancyfoot[LE]{\bfseries \thepage}
\renewcommand{\headrulewidth}{0.3pt}
\renewcommand{\footrulewidth}{0pt}

\fancypagestyle{plain}{%
\fancyhf{} % clear all header and footer fields
\fancyfoot[RO]{\bfseries \thepage}
\fancyfoot[LE]{\bfseries \thepage}
\renewcommand{\headrulewidth}{0pt}
\renewcommand{\footrulewidth}{0pt}}

% Encabezado en la parte superior de las hojas
\renewcommand{\sectionmark}[1]{\markright{\thesection.\ #1}}
\renewcommand{\chaptermark}[1]{\markboth{\chaptername\ \thechapter.\ #1}{}}

% Cuadros de observación
\newcommand{\Observacion}[1]{
\begin{center}
\begin{tabular}[c]{|c|}\hline
#1\\ \hline
\end{tabular}
\end{center}
}

\usepackage{color}            % Permite definir colores
\definecolor{celeste}{rgb}{0.90,0.90,0.95}
\definecolor{verde}{rgb}{0.85,1.00,0.85}
\definecolor{amarillo}{rgb}{1.00,1.00,0.85}

% Caja temporal para guardar el texto - Que se usa en las cajas de texto
\newsavebox{\temporalbox}

% Cuadros de Sabias qué - Icono: lamparita
\newenvironment{sabias_que}
 {\begin{lrbox}{\temporalbox}\begin{minipage}{0.95\textwidth}
  \begin{minipage}{0.05\textwidth}\includegraphics[height=4ex]{graficos/lamparita}\end{minipage}
  \begin{minipage}{0.80\textwidth}\bf Sabías que \ldots\end{minipage}\\
  \begin{small}}
 {\end{small}\end{minipage}\end{lrbox}
  \begin{figure*}[ht]
  \centering
  \fboxsep 0.5em
  \fcolorbox{black}{celeste}{
  \usebox{\temporalbox}}
  \end{figure*}}

% Cuadros de Atención - Icono: Triángulo con !
\newenvironment{atencion}
 {\begin{lrbox}{\temporalbox}\begin{minipage}{0.95\textwidth}
  \begin{minipage}{0.05\textwidth}\includegraphics[height=3ex]{graficos/atencion}\end{minipage}
  \begin{minipage}{0.80\textwidth}\bf Atención \end{minipage}\vspace{0.8ex}\\
  \begin{small}}
 {\end{small}\end{minipage}\end{lrbox}
  \begin{figure*}[ht]
  \centering
  \fboxsep 0.5em
  \fcolorbox{black}{amarillo}{
  \usebox{\temporalbox}}
  \end{figure*}}

\usepackage{framed}           % Cuadros multihoja con color de fondo

% Cuadros de referencia del lenguaje. - Icono: logo python
\newenvironment{referencia_python}
 {\definecolor{shadecolor}{rgb}{0.85,1.00,0.85}%
  \begin{shaded}%
  \begin{minipage}{0.80\textwidth}\hspace{-1em}\bf Referencia del lenguaje Python\end{minipage}%
  \hfill%
  \begin{minipage}{0.05\textwidth}\includegraphics[height=4ex]{graficos/logo-python}\end{minipage}\\
  \rule[1.00ex]{1.0\textwidth}{0.01ex}\\
  \vspace{-2ex}%
  \begin{small}}
 {\end{small}\end{shaded}}

\newenvironment{sintaxis}[1]
  {\hspace{-1em}\textbf{#1}%
   \vspace{1ex}\\
   \begin{minipage}{0.05\textwidth}\end{minipage}\hfill\begin{minipage}{0.90\textwidth}}
  {\end{minipage}\hfill\begin{minipage}{0.05\textwidth}\end{minipage}\\}

% Parametros para los listings de python
\lstset{language=C,
    numbers=left,
    numberstyle=\tiny,
    numbersep=5pt,
    showstringspaces=false,
    basicstyle=\ttfamily,
    inputencoding=utf8,
    literate=%
        {á}{{\'a}}1
        {é}{{\'e}}1
        {í}{{\'i}}1
        {ó}{{\'o}}1
        {ú}{{\'u}}1
        {ü}{{\"u}}1
        {Á}{{\'A}}1
        {É}{{\'E}}1
        {Í}{{\'I}}1
        {Ó}{{\'O}}1
        {Ú}{{\'U}}1
        {Ü}{{\"U}}1
        {ñ}{{\~n}}1
        {Ñ}{{\~N}}1
        {¡}{{!`}}1
        {¿}{{?`}}1
}
%    basicstyle=\ttfamily\normalsize,
%    keywordstyle=\ttfamily\bfseries,
%    identifierstyle=\ttfamily,
%    commentstyle=\ttfamily\itshape,
%    stringstyle=\ttfamily,

\usepackage{float}            % Agrega estilos a los floats

\floatstyle{ruled}
\newfloat{codigo-float}{thp}{lop}
\floatname{codigo-float}{Código}

\newenvironment{codigo}[2]{\begin{codigo-float}
    \caption{{\small\texttt{#1}}\textbf{:}\ \small #2}
    }{\end{codigo-float}}

\lstnewenvironment{codigo-c}{\lstset{xleftmargin=10pt,tabsize=4}}{}
\lstnewenvironment{codigo-c-plano}{\lstset{xleftmargin=10pt,tabsize=4,numbers=none}}{}

% Numeracion para secciones y subsecciones - Hace falta?
%\renewcommand{\thesection}{\thechapter.\arabic{section}}
%\renewcommand{\thesubsection}{\thesection.\arabic{subsection}}

% Estilos usados en el texto
\theoremstyle{definition}
\newtheorem{definicion}{Definici\'on}[section]
\newtheorem{ejemplo}{Ejemplo}[section]
\newtheorem{problema}{Problema}[section]
\newtheorem{ejercicio}{Ejercicio}[section]
\newtheorem{ejerciciof}{}[section]
\newtheorem{problemac}{Problema}[chapter]
\newtheorem{ejercicioc}{Ejercicio}[chapter]
\renewcommand{\qedsymbol}{} % El proof inserta dos qed al final
\newenvironment{solucion}[1][Solución]{\begin{proof}[#1]}{\end{proof}}

\theoremstyle{remark}
\newtheorem{observacion}{Observaci\'on}[section]
\newtheorem{nota}{Nota}[section]

% Traducciones propias
\addto\captionsspanish{
\renewcommand{\chaptername}{Unidad}
\renewcommand{\contentsname }{Contenidos }
}

% Definiciones varias - En uso
\def\esp{\hspace{0.5ex}}
\def\ra{\rightarrow}
\def\qq{\textquoteright}

\begin{document}

% Carat no esta.
%\include{carat}
%\title{Algoritmos y Programación II \\ Con lenguaje C}
%\maketitle

%\tableofcontents  %Comentar  para no imprimir la toc

% Comando para configurar el número de capítulo
%\setcounter{chapter}{11}

% Copyright (C) 2010-2013, Maximiliano Curia <maxy@gnuservers.com.ar>,
%               2010-2013, Margarita Manterola <marga@marga.com.ar>

% Esta obra está licenciada de forma dual, bajo las licencias Creative
% Commons:
%  * Atribución-Compartir Obras Derivadas Igual 2.5 Argentina
%    http://creativecommons.org/licenses/by-sa/2.5/ar/
%  * Atribución-Compartir Obras Derivadas Igual 3.0 Unported
%    http://creativecommons.org/licenses/by-sa/3.0/deed.es_AR.
%
% A su criterio, puede utilizar una u otra licencia, o las dos.
% Para ver una copia de las licencias, puede visitar los sitios
% mencionados, o enviar una carta a Creative Commons,
% 171 Second Street, Suite 300, San Francisco, California, 94105, USA.

\renewcommand{\chaptermark}[1]{\markboth{#1}{}}
\renewcommand{\thesection}{\arabic{section}}
\chapter*{El lenguaje C}

En este apunte se dará una introducción básica al lenguaje de programación C,
asumiendo un conocimiento previo de técnicas de programación en algún otro
lenguaje.

\section{Características básicas del lenguaje}

Se podría decir que el lenguaje de programación C es un lenguaje
\textit{sencillo}, fácil de cubrir en poco tiempo, ya que tiene pocas palabras
reservadas, y una biblioteca estándar más acotada que la de otros lenguajes.

Sin embargo, la especificación actual contiene 701 páginas
\footnote{http://www.open-std.org/jtc1/sc22/wg14/www/docs/n1570.pdf} y es
posible crear código extremadamento \textit{ofuscado}
\footnote{http://www.ioccc.org/}, de modo que no es realmente correcto decir
que es sencillo. \\

C es un lenguaje de programación estructurado, de medio nivel, y muy portable.
Esto se debe a que el modelo de computadora que usa el lenguaje se puede
ajustar a una gran variedad de equipos. A veces se lo considera como un
lenguaje ensamblador de alto nivel, ya que el programador suele tener que
tener en cuenta detalles sobre cómo se representan los elementos del programa en
la máquina, o manejar (pedir y liberar) los recursos del sistema desde el
código.

\section{Estándares del lenguaje}

A lo largo de la historia se han desarrollado tres estándares principales.
\footnote{El último estándar del lenguaje es C11, publicado en el 2011, sin
embargo todavía no tuvo un impacto notable, por lo que nos concentraremos en
C99.}

\begin{description}
\item[K\&R] El estándar publicado en la primera edición del libro
''El lenguaje de programación C'' de Kernighan y Ritchie.
\item[C89] Publicado en la referencia estándar ANSI X3.159-1989 y luego en el
estándar ISO/IEC 9899:1990, así como en la segunda edición del mismo libro.
\item[C99] El estándar ISO, publicado en 1999.
\end{description}

Si bien a esta altura la mayoría de los compiladores de C soportan
prácticamente el estándar completo de C99
\footnote{http://gcc.gnu.org/c99status.html}, una gran parte de código
disponible utiliza todavía el estándar C89; es por eso que en este apunte se
hace especial distinción con aquellos detalles que pertencen al estándar C99.

\section{Tipos básicos}

C cuenta con una variedad de tipos numéricos.  De estos, los tipos enteros
pueden tomar los modificadores \lstinline!signed! o \lstinline!unsigned! para
indicar si son o no signados.

A continuación una tabla con los distintos tipos de C, ordenados según el
espacio que ocupan en memoria.

\begin{description}
\item[bool]
 Se agregó en C99. Puede contener los valores 0 y 1. Incluyendo la biblioteca
\lstinline!<stdbool.h>!, se pueden utilizar los valores \lstinline!true! y
\lstinline!false! (equivalentes a 1 y 0 respectivamente).

\item[char]
 Tipo entero, por omisión \lstinline!unsigned!, de tamaño de 1 byte.

\item[short]
 Tipo entero, por omisión \lstinline!signed!, debe ocupar menos espacio o el
mismo que int. En el compilador gcc, arquitectura Intel 32 bits, mide 16 bits.

\item[int]
 Tipo entero, por omisión \lstinline!signed!, es el tipo \textit{natural} de la
 arquitectura. En el compilador gcc, arquitectura Intel 32 bits, mide 32 bits.

\item[long]
 Tipo entero, por omisión \lstinline!signed!, debe ocupar igual o más espacio
que long. En el compilador gcc, arquitectura Intel 32 bits, mide 32 bits.

\item[long long]
 Tipo entero, por omisión signed, debe ocupar igual o más espacio
 que long. En el compilador gcc, arquitectura Intel 32 bits, mide 64 bits.

\item[float]
 Tipo real, cumple con el estándar IEEE 754 de simple precisión (32 bits).

\item[double]
 Tipo real, cumple con el estándar IEEE 754 de doble precisión (64 bits).

\item[long double]
 Tipo real, según la arquitectura y las opciones de compilación, puede cumplir
con el estándar IEEE 754 de doble precisión (64 bits) o de doble precisión
extendida (más de 79 bits, 80 bits en arquitecturas Intel 32 bits).

\item[complex]
 Se agregó en C99, representa un número complejo.  Ocupa dos
\lstinline!doubles!, y requiere incluir \lstinline!<complex.h>!

\item[float complex]
 Se agregó en C99, de menor tamaño que el complejo
común. Ocupa dos floats, también requiere \lstinline!<complex.h>!.

\item[long complex]
 Se agregó en C99, ocupa dos \lstinline!long doubles!, y también requiere
\lstinline!<complex.h>!.

\item[void]
 No se puede usar como un tipo de una variable, se usa
 para señalar que una función no devuelve nada o no recibe nada.
\end{description}

Además, se puede utilizar el modificador de \lstinline!const! para declarar
una variable que puede inicializarse pero una vez inicializada no puede
modificarse.

La inicialización de las variables se realiza cuando se definen. En el caso
de las funciones, los valores que reciben los parámetros actuan como
inicializadores.

\section{Sintáxis básica}

Asumiendo conocimientos previos de programación, se describe a continuación la
sintaxis básica del lenguaje de programación C.

\subsection{Instrucciones}

Las instrucciónes en C son lo que forman las secuencias que ejecutarán los
programas, las instrucciones terminan en \lstinline!;! y donde puede haber una
instrucción puede haber también una serie de instrucciones entre llaves:
\lstinline!{! (para comenzar el bloque) y \lstinline!}! (para terminar el
bloque).

\subsection{Valores literales}

Los valores literales son valores explícitamentes escritos en el código.
Y merecen un breve comentario en este resumen.

Los valores númericos se pueden escribir en decimal (\lstinline!4095!), en
octal (\lstinline!07777!) o en hexadecimal (\lstinline!0xFFF!). Además, se les
puede agregar al final una \lstinline!L! para indicar que es un
\lstinline!long! o una \lstinline!U! para indicar que es un valor
\lstinline!unsigned!.

En el caso de los valores reales, se los puede representar con punto como
separador entre parte entera y decimal o en notación científica. Por omisión,
estos valores serán de tipo \lstinline!double!, pero se puede usar una letra
\lstinline!F! como sufijo del valor para que se los tome como
\lstinline!float!.

Los caracteres también son valores numéricos, pero se pueden escribir a través
del símbolo que representan escribiéndolos entre comillas simples. Asumiendo
que se utiliza un sistema en ASCII, \lstinline!'A'! será lo mismo que escribir
el valor 65.  Varios de los caracteres especiales (como el fin de línea) se
pueden representar en C como una secuencia de \lstinline!\!  seguida de algún
caracter, por ejemplo, el fin de línea se representa como \lstinline!'\n'!.

A continuación una tabla con las secuecias que representan caracteres
especiales.

\begin{tabular}{ccl}
Secuencia & Nombre & Descripción \\
\hline
\lstinline!\n! &  NL & fin de línea (enter) \\
\lstinline!\t! &  HT & tabulación horizontal (tab) \\
\lstinline!\v! &  VT & tabulación vertical \\
\lstinline!\b! &  BS & retroceso (backspace) \\
\lstinline!\r! &  CR & retorno de carro \\
\lstinline!\f! &  FF & avance de hoja \\
\lstinline!\a! &  BEL & señal audible (beep) \\
\lstinline!\\! &  \verb!\! & contra barra \\
\lstinline!\'! &  \verb!'! & comillas simples \\
\lstinline!\0! &  NUL & caracter nulo \\
\lstinline!\ooo! & ooo & caracter con el valor octal ooo \\
\lstinline!\xHH! & HH  & caracter con el valor hexadecimal HH
\end{tabular}

Una cadena literal en C se escribe dentro de comillas dobles, por ejemplo
\lstinline!"ejemplo"! será un vector de 8 \lstinline!char!, el último de estos
caracteres será \lstinline!'\0'! (un caracter con valor entero 0).

\subsection{Estructuras Condicionales}

La estructura condicional evalúa la condición, si es verdadera ejecuta el
bloque verdadero, sino ejecuta el bloque alternativo.

En C, el condicional tiene dos formas básicas:

\begin{codigo-c-plano}
if (condición) {
    instrucciones;
}
\end{codigo-c-plano}

En este caso, el bloque se ejecuta únicamente si es verdadero y si no lo es,
no se ejecuta nada.  La otra opción es:

\begin{codigo-c-plano}
if ( condición ) {
    instrucciones-verdadero;
} else {
    instrucciones-falso;
}
\end{codigo-c-plano}

En ambos casos, cuando se trate de una única instrucción pueden omitirse las
llaves, pero en general se recomienda utilizarlas de todas maneras para
prevenir errores si luego se arreglan más instrucciones.

Una forma alternativa de la estructura condicional es la de múltiples
condiciones anidadas, que suele escribirse:

\begin{codigo-c-plano}
if ( condición_1 ) {
    cuerpo_1;
} else if ( condición_2 ) {
    cuerpo_2;
} else if ( condición_3 ) {
    cuerpo_3;
} else if ( condición_4 ) {
    cuerpo_4;
} else {
    cuerpo_else;
}
\end{codigo-c-plano}

Este tipo de estructura verifica las condiciones en cascada, hasta que una de
ellas sea verdadera y en ese caso se ejecutará el cuerpo correspondiente; de
no ser así, llegará al \lstinline!else! final.  Se trata únicamente de una
forma de escribir cómodamente los condicionales anidados.

Otra estructura de selección multiple es el \lstinline!switch!, que se muestra a
continuación.

\begin{codigo-c-plano}
switch ( expresión_entera ) {
case valor_entero_1:
    instrucciones;
    break;
case valor_entero_2:
    instrucciones;
    break;
...
default:
    instrucciones;
    break;
}
\end{codigo-c-plano}

En este caso, se compara la \lstinline!expresion_entera! con los distintos
valores enteros, y cuando coincide, se ejecutan las correspondientes
instrucciones.  De omitirse la instrucción \lstinline!break!, se continúa
ejecutando el siguiente bloque, sin importar que corresponda a otro valor.  En
el caso en que no coincida con ninguno de los valores, se ejecutará el bloque
\lstinline!default!.

Es importante notar que este tipo de selección multiple sólo puede operar con
enteros, de manera que tanto la expresión usada con la instrucción
\lstinline!switch! como cada uno de los posibles valores usados con
\lstinline!case! son tomados como enteros para compararlos.

\subsubsection{Concepto de verdadero}

El concepto de verdadero de C es \textit{todo lo que es 0 es falso, todo lo
demás en verdadero}.

En C99 existe el tipo \lstinline!bool! que es 0 en el caso de falso, y 1 en
caso de verdadero, pero no es necesario utilizar este tipo para las
condiciones, cualquier variable que valga 0 se considerará falsa, y cualquier
variable con un valor distinto de 0 se considerará verdadera.

\subsubsection{Operadores de comparación}

En C existen diversos operadores de comparación entre valores, a continuación
una tabla con los operadores más comunes.

\begin{tabular}{cl}
Operador & Significado \\
\hline
\lstinline!a1 == a2! & \lstinline!a1! vale lo mismo que \lstinline!a2! \\
\lstinline|a1 != a2| & \lstinline!a1! no vale lo mismo que \lstinline!a2! \\
\lstinline!a1 > a2! & \lstinline!a1! es mayor que \lstinline!a2! \\
\lstinline!a1 < a2! & \lstinline!a1! es menor que \lstinline!a2! \\
\lstinline!a1 >= a2! & \lstinline!a1! es mayor o igual que \lstinline!a2! \\
\lstinline!a1 <= a2! & \lstinline!a1! es menor o igual que \lstinline!a2! \\
\end{tabular}

Además, los operadores de comparación pueden unirse o modificarse para formar
expresiones más complejas.

\begin{tabular}{cl}
Operador & Significado \\
\hline
\lstinline!e1 && e2! & Debe cumplirse tanto \lstinline!e1! como \lstinline!e2! \\
\lstinline!e1 || e2! & Debe cumplirse \lstinline!e1!, \lstinline!e2! o ambas \\
\lstinline?! e1?     & \lstinline!e1! debe ser falso \\
\end{tabular}

Incluyendo la biblioteca \lstinline!<iso646.h>! se puede usar las palabras
\lstinline!and!, \lstinline!or!, \lstinline!not!, y otros, como operadores, de
la misma manera que son operadores en otros lenguajes.

\subsection{Ciclos}

El bucle \textit{mientras} en C tiene la siguiente forma:

\begin{codigo-c-plano}
while ( condición ) {
    cuerpo;
}
\end{codigo-c-plano}

La condición es evaluada en cada iteración, y mientras sea verdadera se
ejecuta el cuerpo del bucle.

También existe un bucle \lstinline!do...while!:

\begin{codigo-c-plano}
do {
    cuerpo;
} while ( condición );
\end{codigo-c-plano}

La diferencia con el anterior es que asegura que cuerpo va a ejecutarse al
menos una vez, ya que la condición se evalúa después de haber ejecutado el
cuerpo.

El lenguaje C cuenta con un bucle iterativo \textit{for}, un poco distinto a
otros bucles del mismo nombre.  Para comprenderlo mejor es importante notar
que las dos siguientes porciones de código son equivalentes:

\begin{codigo-c-plano}
for (inicialización; condición; incremento) {
    cuerpo;
}
\end{codigo-c-plano}

\begin{codigo-c-plano}
inicialización;
while (condición) {
    { cuerpo; }
    incremento;
}
\end{codigo-c-plano}

\subsection{Variables}

Todas las variables en C hay que declararlas antes de poder usarlas, la
declaración se hace de la siguiente manera:

\begin{codigo-c-plano}
tipo nombre_variable;
\end{codigo-c-plano}

Se pueden declarar varias variables del mismo tipo separandolas con comas.

\begin{codigo-c-plano}
tipo nombre_variable_1, nombre_variable_2;
\end{codigo-c-plano}

Además, es posible asignar un valor de inicialización al declararlas:

\begin{codigo-c-plano}
tipo nombre_variable_1 = valor_1 , nombre_variable_2 = valor 2;
\end{codigo-c-plano}

\subsection{Comentarios}

En C89 la única forma de poner comentarios es utilizando bloques que comiencen
con \lstinline!/*! y terminen con \lstinline!*/!. En C99, además, se agregó
soporte de comentarios \textit{hasta el final de la línea}, estos empiezan con
\lstinline!//!.

\subsection{Funciones}

Las funciones en C se definen de la siguiente manera:

\begin{codigo-c-plano}
tipo funcion (tipo_1 argumento_1, ..., tipo_n argumento_n)
{
    intrucciones;
    ...;
    return valor_retorno;
}
\end{codigo-c-plano}

Es decir que el tipo que devuelve la función se coloca antes del nombre de la
función, y luego se colocan los argumentos que recibe la función, precedidos
por su tipo.  En el caso de no recibir ningún argumento, se puede colocar
simplemente \lstinline!()! o \lstinline!(void)!.

El cuerpo de las funciones contendrá una secuencia de declaración de
variables, instrucciones, bloques, estructuras de control, etc.  \\

Una función debe estar declarada antes (leyendo el archivo desde arriba hacia
abajo) de poder llamarla en el código.  Es por esto que la definición (o
prototipo) de la función puede colocarse antes del contenido de la función, de
forma que pueda ser utilizada por funciones que se encuentran implementadas
antes.  En ese caso será:

\begin{codigo-c-plano}
tipo funcion (tipo_1 argumento_1, ..., tipo_n argumento_n);
\end{codigo-c-plano}

\subsection{Punto de entrada}

Se llama punto de entrada a la porción de código que se ejecuta en primer
lugar cuando se llama al programa desde la línea de comandos. En C el punto de
entrada es la función \lstinline!main! y dado que es una función que
interactúa con el sistema, tiene un prototipo en particular (con dos
opciones):

\begin{codigo-c-plano}
int main (void);
\end{codigo-c-plano}

Se puede ver que la función main devuelve un entero, que será el valor de
retorno del programa, 0 indicará que el programa se ejecutó exitosamente y
cualquier otro valor indicará un error. Esta opción, que no recibe parámetros,
se utiliza cuando no se quieren tener en cuenta los parámetros de línea de
comandos.  La otra opción se utiliza cuando sí se quieren tener en cuenta
estos parámetros:

\begin{codigo-c-plano}
int main (int argc, char *argv[]);
\end{codigo-c-plano}

En este caso, los parámetros \lstinline!argc! y \lstinline!argv! podrían tener
cualquier otro nombre, pero es convención usar estos dos. Su significado es
\emph{la cantidad de argumentos} y \emph{un vector de punteros a los
argumentos} respectivamente.  Más adelante se verán en detalle los temas de
vectores y punteros.


\section{Tipos derivados}

\subsection{Vectores}

Los vectores (o arreglos) son bloques continuos de memoria que contienen un
número de elementos del mismo tipo. Se los declara de la siguiente manera:

\begin{codigo-c-plano}
tipo_elemento nombre_vector[tamaño];
\end{codigo-c-plano}

Opcionalmente se puede inicializar el contenido:

\begin{codigo-c-plano}
tipo_elemento nombre_vector[] = { valor_0, valor_1, ... valor_n-1 };
\end{codigo-c-plano}

En este caso el tamaño es implícito, el compilador lo decide a partir de la
cantidad de elementos ingresada en el inicializador. \\

Para acceder al contenido de un vector se utiliza a través del índice del
elemento dentro del vector. Los índices del vector van desde 0 hasta
\lstinline!largo-1!.  Es importante recordar que \lstinline!vector[largo]! es
una posición inválida dentro del vector. Es decir:

\begin{codigo-c-plano}
tipo vector[largo];
vector[0] = valor; // asigna valor al primer elemento
valor = vector[9]; // toma el valor del décimo elemento
vector[largo-1] = valor; // asigna valor al último elemento
\end{codigo-c-plano}

Si se accede a un vector por su nombre, sin ningún índice, se obtiene
la posición en memoria del vector. Esto es una optimización para evitar tener
que hacer copias de (posiblemente) grandes bloques de memoria al llamar a una
función que recibe un vector. Esto tiene varias consecuencias:

\begin{itemize}
\item Los vectores se pasan como referencia, ya que lo que se pasa es la
posición de memoria donde se encuentra el vector.

\item Al recibir un vector en una función no hace falta definir el largo de
este, ya que el tamaño en memoria debería haber sido definido previamente.
\end{itemize}

Esto hace que en ciertas situaciones un vector tenga un comportamiento similar
al de los punteros, aunque no exactamente igual.

\subsection{Punteros}

Los punteros son direcciones de memoria. En C los punteros requieren tener un
tipo asociado, según el tipo de datos al que apuntan (es decir, el tipo de
datos que se encuentra en la porción de memoria indicada por el puntero).

El tipo \lstinline!void*! se usa para apuntar a posiciones de memoria que
contengan un dato de tipo desconocido.

La declaración de un puntero es igual que para una variable normal, pero se le
agrega un \lstinline!*! delante. Es decir:

\begin{codigo-c-plano}
tipo *puntero_a_tipo;
\end{codigo-c-plano}

Nota: el lenguaje permite escribir el \lstinline!*! pegado al tipo, también:

\begin{codigo-c-plano}
tipo* puntero_a_tipo;
\end{codigo-c-plano}

Sin embargo las siguientes líneas son equivalentes:

\begin{codigo-c-plano}
tipo *puntero, variable;
tipo* puntero, variable;
\end{codigo-c-plano}

En ambos casos sólo la primera variable es declarada como un puntero, la
segunda es sólo una variable del tipo \lstinline!tipo!.

Vale la pena aclarar que al declarar un puntero este no se inicializa con
ningún valor determinado (contiene \textit{basura}), ni se crea un espacio en
memoria capaz de contener un valor de tipo \lstinline!tipo!, por lo que se le
debe asignar una dirección de memoria válida antes de poder operar con este.

Para obtener la dirección de memoria de un valor ya creado se utiliza el
operador \lstinline!&!:

\begin{codigo-c-plano}
puntero = &variable;
\end{codigo-c-plano}

La operación contraria (\textit{desreferenciar} un puntero) es \lstinline!*!,
que accede al valor referenciado por una dirección de memoria:

\begin{codigo-c-plano}
*puntero = valor;
\end{codigo-c-plano}

Dado que en C la mayoría de las variables pasan por valor (incluyendo los
punteros y con la única excepción de los vectores), si se pasa el valor de una
dirección de memoria (un puntero) es posible modificar el valor referenciado
por esa dirección. Por ejemplo, para leer un entero usando scanf se debe hacer:

\begin{codigo-c-plano}
scanf("%d",&entero);
\end{codigo-c-plano}

\subsection{Conversión forzada de tipos (\textit{cast})}

La conversión forzada, o \textit{casteo}, se utiliza para convertir un valor
de un tipo a otro, cuando el compilador no es capaz de hacerlo
automáticamente.  Se lo logra anteponiendo un tipo entre paréntesis delante de
una expresión. Por ejemplo:

\begin{codigo-c-plano}
double resultado = 3 / 2; // división entera
                          // resultado = 1.0
double resultado = (double) 3 / 2 // división flotante
                                  // resultado = 1.5
\end{codigo-c-plano}

\subsection{Estructuras}

Las estructuras permiten combinar distintos tipos de datos en un mismo bloque,
de la siguiente forma:

\begin{codigo-c-plano}
struct estructura {
    tipo_0 atributo_0;
    tipo_1 atributo_1;
    ...
    tipo_n atributo_n;
};
\end{codigo-c-plano}

Esta porción de código define un nuevo tipo de datos, llamado
\lstinline!struct estructura!, que se puede utilizar en el resto del
código.

Es importante notar que este código lleva un \lstinline!;!, es uno de los
pocos casos en los que debe escribirse un \lstinline!;! luego de una
\lstinline!}!, y una fuente muy común de errores.

Las estructuras se declaran al nivel de declaraciones, (donde se
definen prototipos de funciones, se incluyen encabezados, se definen enum,
etc).

Una estructura ocupa en memoria por lo menos la suma de cada uno de sus
atributos, además, puede haber una porción de memoria desperdiciada en la
\textit{alineación} de los datos. \\

Para acceder a los elementos de una estructura se utiliza el operador
\lstinline!.!, por ejemplo:

\begin{codigo-c-plano}
struct prueba {
    char nombre[10];
    int valor;
};
...
    struct prueba ejemplo;
    ejemplo.valor = 0;
...
\end{codigo-c-plano}

Como todos los otros tipos de datos excepto los vectores, las estructuras en C
se pasan por valor. Al trabajar con estructuras, casi siempre se utilizan
punteros para pasarlas a las funciones, para evitar crear grandes copias en
memoria, y para poder modificar sus atributos.  Para acceder a un elemento, en
ese caso, se puede escribir:

\begin{codigo-c-plano}
(*puntero_estructura).nombre
\end{codigo-c-plano}

Como se trata de una operación muy común, esto mismo se puede escribir
\footnote{Esta pequeña facilidad es un poco de \textit{azúcar sintáctico} del
lenguaje}:

\begin{codigo-c-plano}
puntero_estructura->nombre
\end{codigo-c-plano}

\subsection{Renombrado de tipos}

El operador \lstinline!typedef! se utiliza para darle un nuevo nombre a un
tipo existente, con la siguiente sintaxis.

\begin{codigo-c-plano}
typedef viejo_tipo nuevo_tipo;
\end{codigo-c-plano}

Se puele utilizar \lstinline!typedef! para darle un nuevo nombre a la
estructura, de forma que no haga falta anteponer
\lstinline!struct! para usarlo, esto es:

\begin{codigo-c-plano}
typedef struct _estructura {
    tipo_1 nombre_1;
    tipo_2 nombre_2; } estructura;
\end{codigo-c-plano}

Una vez definido de esta manera, se utiliza simplemente
\lstinline!estructura variable;! para declarar una variable del tipo.

En el \lstinline!typedef! el nombre intermedio \lstinline!_estructura! puede
omitirse, pero será necesario cuando una estructura haga referencia a si misma
dentro de su declaración.

\subsection{Valores Enumerados}

Es posible definir enumeraciones de valores enteros mediante el tipo
\lstinline!enum!.

\begin{codigo-c-plano}
enum dias_semana {DOMINGO, LUNES, MARTES, MIERCOLES, JUEVES, VIERNES,
                  SABADO};
enum {TRUE=1, FALSE=0, MAX_LARGO=1024};
\end{codigo-c-plano}

En este ejemplo se define un tipo \lstinline!enum dias_semana!, que define los
valores \lstinline!DOMINGO=0!, \lstinline!LUNES=1! y así sucesivamente. En el
segundo uso de enum no se define un tipo, simplemente se definen valores.

En el código se pueden usar los nombres de esos valores en lugar del valor
en sí. Es una de las formas de \textit{parametrizar} el código.

\subsection{Uniones}

Las uniones son similares a las estructuras, pero en este caso cada elemento
comparte la misma ubicación en memoria.

No son muy utilizadas, pero normalmente se las usa cuando se necesita guardar
un valor de distintos tipos y cada formato es excluyente (sólo uno de los
tipos de datos sirve en cada caso):

\begin{codigo-c-plano}
typedef enum {ENTERO, FLOTANTE} tipo_t;

union contenedor {
    int entero;
    float flotante;
}

struct uso_prueba {
    tipo_t tipo;
    union contenedor valor;
}
\end{codigo-c-plano}

O cuando se quiere poder tener dos forma se acceder a los mismos datos.

\subsection{Asignación y Comparación}

En C las asignaciones y comparaciones pueden utilizarse en cualquier parte de
código, como cualquier otra expresión. Lo cual da lugar a errores, como por
ejemplo un error usual:

\begin{codigo-c-plano}
while (c = 1) {
    ... // código que eventualmente modifica el valor de c
}
\end{codigo-c-plano}

Lo que hace que un error simple se convierta en un bucle infinito en tiempo de
ejecución. \\

Además de la asignación normal:

\begin{codigo-c-plano}
e = f // asigna el valor de f a e
\end{codigo-c-plano}

Es C también es válido utilizar var op= valor, para obtener var = var op
valor, ejemplos:

\begin{codigo-c-plano}
e += f // e = e + f
e -= f // e = e - f
e *= f // e = e * f
\end{codigo-c-plano}

Además, cuando se debe incrementar o decrementar un valor en 1, C provee
pre/post (in/de)crementos, por ejemplo:

\begin{codigo-c-plano}
a = 0; b = 0; c = 0; d = 0;
e = a++; // a = a + 1, Post incremento, e = 0, a = 1
e = ++b; // b = b + 1, Pre incremento, e = 1, b = 1
e = c--; // c = c - 1, Post decremento, e = 0, c = -1
e = --d; // d = d - 1, Pre  decremento, e = -1, d = -1
\end{codigo-c-plano}

Las expresiones en C propagan valores de izquierda a derecha, el valor que se
propaga es el que puede ser revisado eventualmente por las estructuras
\lstinline!while!, \lstinline!if!, \lstinline!for!, etc.

Ejemplo:

\begin{codigo-c-plano}
a = b = c = d = e = f = 1; // usa la propagación para asignar varias
                           // variables a la vez.
\end{codigo-c-plano}

\subsection{Constantes}

C tiene tres tipos de constantes distintos: las que se definen con el
preprocesador, los tipos enumeraros y las variables con el modificador const.

Las constantes del preprocesador de C son \textit{macros} que son reemplazados
por el preprocesador, que corre antes que el compilador. El preprocesador no
conoce el lenguaje, sólo busca ocurrencias de una secuencia de caracteres y
las reemplaza por otras, lo cual puede ser problemático en algunas situaciones
particulares.  Ejemplos:

\begin{codigo-c-plano}
#define MAX_LARGO 2048
#define AUTHOR "Mi Nombre"
#define DATE "2009-09-01"
#define LICENSE "CC-3.0-BY-SA"
...
    int vector[MAX_LARGO];
\end{codigo-c-plano} 

Los valores enumerados que ya fueron mencionados anteriormente sólo pueden
contener valores enteros (\lstinline!int!), es la forma recomendada de tener
constantes enteras, ya que es fácilmente parametrizable y no tiene las
desventajas del preprocesador.

Las variables con el modificador const pueden usarse y una vez inicializadas
no puede alterarse el contenido sin hacer un casteo.

\section{Ejemplo básico}

Desde hace muchos años este es el ejemplo básico de programación en C.

\begin{codigo-c}
#include <stdio.h>

int main(int argc, char* argv[])
{
    printf("Hola mundo\n");
    return 0;
}
\end{codigo-c}

En la primera línea de este ejemplo hay una instrucción \lstinline!#include!,
se trata de una instrucción al preprocesador\footnote{El preprocesador es una
herramienta que corre al compilar el programa, antes de correr el compilador,
las instrucciones de preprocesador siempre comienzan con \lstinline!#!}.
Esta instrucción significa que todo lo que está en el archivo especificado se
incluye dentro del archivo actual. Los \lstinline!<>! alrededor del nombre del
archivo significan que el preprocesador debe buscar el archivo en la ruta de
inclusión del sistema.  Si se utilizara \lstinline!""! en lugar de
\lstinline!<>!, se buscará el archivo en la ruta actual de compilación.

En la práctica las \lstinline!<>! se utilizan para incluir encabezados
(conjuntos de prototipos, definiciones de tipos y constantes, etc) de las
bibliotecas externas al programa que se vayan a utilizar, que se deben
encontrar instaladas en el sistema. Mientras que las comillas dobles se
utilizan para incluir encabezados propios de otras porciones del mismo
programa.

En particular la biblioteca \lstinline!stdio.h! es la biblioteca estándar de
entrada y salida, en este caso es incluida para poder usar \lstinline!printf!
que es una función de la biblioteca estándar de C para imprimir por salida
estándar (normalmente, la consola).  En este caso, \lstinline!printf! recibe
un único parámetro que será la salida a imprimir; pero puede también recibir
más parámetros, para lograr una salida más avanzada.

El primer argumento de \lstinline!printf! es siempre una cadena, que puede
tener un formato especial o no, indicando qué tipos de variables se deben
imprimir y de qué forma.  Además, puede tener marcas especiales para indicar
el fin de línea (\verb!'\n'!), una tabulación (\verb!'\t'!), una contrabarra
(\verb!'\\'!) y algunos más.

Vale la pena notar que \lstinline!printf! no es parte del lenguaje sino de la
biblioteca estándar, que está especificada en el mismo estándar donde está
especificado el lenguaje pero aún así, no es parte del lenguaje.

Se puede encontrar documentación completa de \lstinline!printf! y de las otras
funciones de biblioteca mediante las páginas del manual, generalmente
instaladas en los sistemas Linux o similares (\verb!$ man 3 printf!) o
mediante el programa gráfico \verb!yelp! en estos mismos sistemas, o bien
on-line en cualquier sitio que publique las páginas de manual en internet
\footnote{http://linux.die.net/man/}
\footnote{http://www.linuxinfor.com/spanish/man3/index.html}.

\section{Compilación}

Para poder compilar programas en C, es necesario contar con un entorno de
programación que permita compilar, enlazar y correr los programas compilados.
Esto requiere tener el compilador de C instalado, junto con la versión para
desarrollar de la biblioteca estándar.

Existen numerosos programas \footnote{Codeblocks, Geany, Anjuta, Kdevelop,
etc} que permiten compilar, enlazar y correr apretando una tecla o eligiendo
una opción desde un menu.  Si bien estos programas son una ayuda para el
desarrollador, no son indispensables, es posible editar el código del programa
con cualquier archivo de texto y luego compilarlo desde la línea de comandos.

El compilador más difundido en los sistemas Linux y uno de los más difundidos
en general es el compilador \textbf{gcc}.  Se trata de un compilador libre,
con muchos años de madurez, y es el que se explica en este apunte.

Asumiendo que el ejemplo presentado antes se grabó como \verb!hola.c!, para
compilarlo usando \verb!gcc! será necesario escribir, en el directorio donde
se encuentra el código del programa:

\begin{verbatim}
$ gcc hola.c -o hola
\end{verbatim}

Esto generará el archivo ejecutable \verb!hola! en ese mismo directorio.  Si
bien no se puede ver en esta sencilla línea de comandos, hay varios pasos
involucrados en la compilación de up programa.

\begin{itemize}
\item En primer lugar, el código es procesado por un \textit{prepocesador},
que se encarga de hacer los \lstinline!#include! antes mencionados, entre
muchas otras cosas.

\item La salida del preprocesador es \textit{compilada}, es decir que el
código C es convertido en código binario que pueda ser ejecutado por la
computadora.

\item Una vez compilado, el programa es \textit{enlazado} con las bibliotecas
que utilizadas, en el ejemplo anterior con la biblioteca estándar de C, para
poder usar \lstinline!printf!.
\end{itemize}

Con \textbf{gcc} es posible realizar estos pasos intermedios uno por uno:

\begin{verbatim}
$ # Prepocesador
$ gcc -E hola.c -o hola.i
$ # Compilador
$ gcc -c hola.i -o hola.o
$ # Enlazador
$ gcc hola.o -o hola
$ # Ejecución del programa
$ ./hola
\end{verbatim}

El compilador gcc tiene una gran variedad de otras opciones que se pueden
consultar en las páginas de manual del mismo (\verb!man gcc!).  A continuación
algunas de las más importantes.

\begin{center}
\begin{tabular}{lp{10cm}}
Opción & Acción \\
\hline
\lstinline!-Wall! &
 Muestra advertencias por cada detalle que el compilador
 detecta como posible error de programación. \\
\lstinline!--pedantic! &
 El compilador se pone en modo pedante, busca más posibles
 errores de programación e interrumpe la compilación por estos. \\
\lstinline!--std=c99! &
 El compilador compila código usando el estándar C99
 (C89 se utiliza por omisión) \\
\lstinline!-g! &
 Pone marcas en el archivo generado para que las use el
 \textit{debugger} (\verb!gdb!). \\
\lstinline!-O! ó \lstinline!-O1! &
 Habilita las optimizaciones básicas. Las optimizaciones pueden cambiar el
flujo del programa por lo que es muy poco recomendable aplicar optimizaciones
sobre un código a utilizar con un debugger. \\
\lstinline!-O2! &
 Habilita todas las optimizaciones basicas y varias avanzadas que se
consideran seguras. \\
\lstinline!-O3! &
 Habilita todas las optimizaciones basicas y varias avanzadas, incluso las que
no se consideran del todo seguras (pueden generar errores en situaciones de
borde). \\
\lstinline!-Os! &
 Habilita las optimizaciones que reducen el tamaño del código. \\
\lstinline!-O0! &
 Deshabilita todas las optimizaciones, este es el comportamiento por omisión. \\
\end{tabular}
\end{center}


% Copyright (C) 2010-2013, Maximiliano Curia <maxy@gnuservers.com.ar>,
%               2010-2013, Margarita Manterola <marga@marga.com.ar>

% Esta obra está licenciada de forma dual, bajo las licencias Creative
% Commons:
%  * Atribución-Compartir Obras Derivadas Igual 2.5 Argentina
%    http://creativecommons.org/licenses/by-sa/2.5/ar/
%  * Atribución-Compartir Obras Derivadas Igual 3.0 Unported
%    http://creativecommons.org/licenses/by-sa/3.0/deed.es_AR.
%
% A su criterio, puede utilizar una u otra licencia, o las dos.
% Para ver una copia de las licencias, puede visitar los sitios
% mencionados, o enviar una carta a Creative Commons,
% 171 Second Street, Suite 300, San Francisco, California, 94105, USA.

\renewcommand{\chaptermark}[1]{\markboth{#1}{}}
\renewcommand{\thesection}{\arabic{section}}
\chapter*{Manejo de memoria en C}

Todas las variables, en el lenguaje C, se definen dentro de alguna función,
fuera de esa función no es posible acceder a ellas.  Al entrar a una función,
a cada una de las variables definidas en esa función se le asigna el espacio
que sea necesario dentro de una pila interna de memoria (\textit{stack}) con
la que cuenta el programa, y al terminar la función se desapila todo lo
definido en ella. Es decir que la pila crece con cada llamado a una función y
decrece con cada función que se termina.

Por otro lado, todos los tipos que C define como parte del lenguaje son de un
tamaño fijo, incluso los definidos por el usuario usando \lstinline!struct!.
Es por eso que el espacio que se reserva en la pila interna de memoria tiene
un tamaño fijo.

Existe, además, otro espacio de memoria que se utiliza cuando el tamaño de los
datos no es fijo.  A este espacio de memoria dinámica se lo llama
\textit{heap}, contiene bloques de memoria, que el programador puede solicitar
para utilizar según sea conveniente.

\section{Obtener el tamaño de un tipo (\textit{sizeof()})}

El operador \lstinline!sizeof()! devuelve el tamaño en bytes de un tipo de
datos (como \lstinline!int!). Por comodidad se le puede pasar tanto el nombre
de una variable o el nombre de un tipo de datos, en ambos casos devolverá el
tamaño del tipo de datos asociado.

\begin{codigo-c-plano}
 int largo, a;
 largo = sizeof(int);  // Cantidad en bytes de int
 largo = sizeof(a);    // Identico a lo anterior
\end{codigo-c-plano}

En este caso, ambas llamadas a \lstinline!sizeof! devuelven el tamaño que
ocupa un entero en memoria en la arquitectura y compilador que se esté
utilizando (por lo general son 4 bytes).

\begin{codigo-c-plano}
 char c;
 largo = sizeof(c); // Cantidad de bytes de un char
\end{codigo-c-plano}

En este caso, se devuelve cuánto ocupa un caracter.  Los caracteres ocupan
siempre 1 byte.

\begin{codigo-c-plano}
 char *puntero;
 largo = sizeof(puntero); // Cantidad de bytes usados por un puntero.
\end{codigo-c-plano}

En este caso, se devuelve la cantidad de bytes que ocupa un puntero. Todos los
punteros tienen el mismo tamaño, y es tal que pueda contener cualquier
dirección de memoria, sea estática o dinámica.

\begin{codigo-c-plano}
 int vector[100];
 largo = sizeof(vector);                // Cantidad de bytes del vector
 largo = sizeof(vector) / sizeof(int);  // Largo del vector (100)
\end{codigo-c-plano}

En este caso, la primera llamada devuelve el tamaño total en bytes ocupado por
el vector (usualmente serían 400 bytes), mientras que la segunda devuelve
siempre 100 sin importar la plataforma, ya que divide el espacio total del
vector por el tamaño de cada uno de los elementos.

\begin{atencion}
Los vectores no siempre se comportan como punteros. En el ejemplo anterior
\lstinline!vector! no se refiere a la posición de memoria donde comienza el
vector, sino a todo el vector. Este es el comportamiento esperado para los
vectores estáticos definidos en el mismo entorno, esto quiere decir, que si el
vector lo recibieramos en una función y le hacemos \lstinline!sizeof(vector)!
el resultado sería la cantidad de bytes usados por un puntero.
\end{atencion}

\section{Memoria dinámica en C}

Mediante la utilización de los punteros, es posible  acceder a cualquier
porción de memoria válida, tanto si se encuentra dentro de la pila interna
como si se encuentra dentro del espacio de memoria dinámica.

Para obtener una porción de memoria válida dentro del espacio de memoria
dinámica, existen en la biblioteca estándar funciones (\lstinline!malloc! y
\lstinline!realloc!) que reservan un bloque de memoria y devuelven su
dirección. Utilizando estas funciones es posible conseguir estructuras de
datos dinámicas, que pueden variar su tamaño según sea necesario, en lugar de
tener un tamaño ya definido.

Es importante notar que la memoria dinámica reservada mediante las funciones
de la biblioteca estándar no es liberada automáticamente, quien haya hecho la
reserva de memoria debe encargarse también de liberarla (con la
correspondiente función de biblioteca estándar, \lstinline!free!), de no ser
así, se dice que el programa \textit{pierde memoria}, ya que los bloques
reservados no pueden volverse a utilizar aún cuando ya no estén en uso.

\subsection{Pedir memoria al sistema (\textit{malloc()})}

Para pedir memoria al sistema se utiliza la función
\lstinline!malloc!\footnote{Para más información: \texttt{man 3 malloc}.},
definida en \lstinline!<stdlib.h>!, cuyo prototipo es el siguiente:

\begin{codigo-c-plano}
void *malloc(size_t tamanio);
\end{codigo-c-plano}

Si el sistema tiene suficiente memoria disponible, \lstinline!malloc! devuelve
un puntero a la primera posición de memoria de un bloque de memoria dinámica,
de \lstinline!tamanio! bytes.

Si ocurriera algún problema, porque el sistema no dispone de suficiente memoria
o similar, la llamada de \lstinline!malloc! devolvería \lstinline!NULL!.

El tipo de datos \lstinline!size_t! es un valor entero sin signo capaz que
contener cualquier tamaño de memoria válido en la arquitectura actual.

\section{Devolver memoria al sistema (\textit{free()})}

Cuando un bloque de memoria ya no es necesario para un programa, se lo debe
devolver al sistema, de forma que el sistema pueda tenerlo nuevamente entre los
recursos a utilizar por otros procesos. La función
\lstinline!free!\footnote{Para más información: \texttt{man 3 free}.} de la
biblioteca estándar hace exactamente esto, su prototipo es:

\begin{codigo-c-plano}
void free(void *puntero);
\end{codigo-c-plano}

Recibe como único parámetro un puntero antes devuelto por \lstinline!malloc! o
\lstinline!realloc!, libera ese bloque de memoria y no devuelve nada. \\

Es importante notar que no es posible liberar una porción de memoria que ya ha
sido liberada.  Esta acción puede provocar el mismo tipo de errores que los
provocados por acceder a una porción de memoria inválida.

\section{Agrandar/achicar un bloque de memoria (\textit{realloc()})}

Cuando se necesita modificar el tamaño de un bloque memoria, se utiliza la
función \lstinline!realloc!\footnote{Para más información: \texttt{man 3
realloc}.} de la biblioteca estándar. Su prototipo es el siguiente:

\begin{codigo-c-plano}
void *realloc(void *puntero_anterior, size_t nuevo_tamanio);
\end{codigo-c-plano}

Recibe un puntero antes obtenido mediante \lstinline!malloc! o
\lstinline!realloc! y el nuevo tamaño del bloque de memoria. Si todo
funciona bien, devuelve un nuevo puntero al nuevo bloque de memoria, copia el
contenido del bloque viejo al nuevo (copia el largo mínimo entre los dos
bloques), y el bloque anterior es liberado.

Si algo falla, devuelve \lstinline!NULL!, y el bloque anterior no se modifica.

Teniendo en cuenta este comportamiento, normalmente \textbf{no} se utiliza una
construcción como la siguiente.

\begin{codigo-c-plano}
/* MAL */
datos = realloc(datos, sizeof(dato_t) * tamanio_nuevo);
\end{codigo-c-plano}

Ya que si \lstinline!realloc! no puede cumplir lo pedido devolverá
\lstinline!NULL! y entonces existirá en memoria un bloque de datos válido, con
información válida, al cual es imposible acceder, ya que se perdió la
dirección de memoria que antes estaba en \lstinline!datos!.

En cambio se debe utilizar:

\begin{codigo-c-plano}
void *aux = realloc(datos, sizeof(dato_t) * tamanio_nuevo);
if ( aux == NULL ) {
    // realloc no pudo pedir el bloque nuevo, hacer algo al respecto.
} else {
    datos = aux;
}
\end{codigo-c-plano}

Otro posible problema a tener en cuenta es cuando se tienen punteros que
apuntan a \textbf{partes} de un bloque de memoria y se llama a
\lstinline!realloc! sobre ese bloque: los punteros pasarán a ser inválidos, ya
que apuntan a direcciones que ya no son las del bloque en cuestión.

De modo que hay que tener mucho cuidado de nunca guardar referencias a
porciones de memoria que pueden ser movidas de lugar mediante
\lstinline!realloc!.

\section{Ejemplo: una pila de tamaño variable}

Tener una pila de tamaño variable es un ejemplo sencillo de manejo de memoria
dinámica. La pila contiene un arreglo de valores, donde se van apilando y
desapilando los elementos; el problema se da cuando se quieren apilar más
elementos que los que la pila puede almacenar, en ese caso se debe reservar un
bloque de memoria de mayor tamaño para que siga siendo posible agregar
elementos a la pila.

En este caso, la estructura de la pila será de la forma:

\begin{codigo-c-plano}
struct _pila {
    void **datos;
    size_t tam;
    size_t largo;
};
typedef struct _pila pila_t;
\end{codigo-c-plano}

Donde \lstinline!cantidad! es la cantidad de elementos almacenada en la pila,
mientras que \lstinline!tamanio! es el tamaño actual de la pila, es decir, la
máxima cantidad de elementos que puede almacenar antes de tener que reservar
una porción mayor de memoria.

\subsection{Creación de la pila}

Cuando se trabaja con memoria dinámica, las funciones de creación de una
estructura, no sólo deben inicializar los atributos de la estructura, sino que
también deben hacer el primer pedido de memoria, para reservar el bloque
inicial con el que se trabajará.

\begin{codigo-c-plano}
pila_t *pila_crear()
{
    // Pedido de memoria para la pila
    pila_t *pila = malloc(sizeof(pila_t));
    if (!pila) return NULL;

    // Pedido de memoria para los datos de la pila
    pila->datos = malloc(MIN_TAM * sizeof(void *));
    if (! pila->datos) {
        free(pila);
        return NULL;
    }

    // Otras inicializaciones
    pila->largo = 0;
    pila->tam = MIN_TAM;
    return pila;
}
\end{codigo-c-plano}

La función \lstinline!malloc! es la encargada de reservar un bloque de memoria
para la estructura de la pila y luego para el bloque de datos que contendrá la
pila. El tamaño de la estructura pila lo obtenemos con
\lstinline!sizeof(pila_t)!, mientras que el tamaño para el bloque inicial de
datos es el resultado de multiplicar una constante \lstinline!MIN_TAM! por el tamaño del tipo de
dato que contiene la pila (\lstinline!void *!, es una pila de
punteros), es decir que en primera instancia la pila podrá
hasta contener \lstinline!MIN_TAM! elementos.

Si por algún motivo el sistema operativo no pudiera reservar la memoria
requerida, la función \lstinline!malloc! devuelve \lstinline!NULL!. En ese
caso, la función de creación de la pila devuelve \lstinline!NULL! para
indicar que no se ha podido crear la pila, si falla el segundo malloc, debemos
liberar el bloque de memoria pedido por el primero malloc.

\subsection{Incremento del tamaño de la pila}

En el caso de agotarse el lugar, será necesario reservar una porción mayor de
memoria.  Es decir que la función \lstinline!apilar! deberá ser de la forma:

\begin{codigo-c-plano}
bool pila_apilar(pila_t *pila, void* valor)
{
    if (!pila) return false;

    // Verifica si hay que agrandar, si no puede devuelve false.
    if (pila->largo == pila->tam) {
        if (! pila_redimensionar(pila, 2 * pila->tam)) {
            return false;
        }
    }

    // Asigna el valor y avanza
    pila->datos[pila->largo] = valor;
    pila->largo++;
    return true;
}
\end{codigo-c-plano}

En esta función, cuando la cantidad de elementos es igual o mayor al tamaño
actual de la pila, se llama a la función \lstinline!pila_redimensionar!, que
será la encargada de reservar un bloque de mayor tamaño, en este caso se le
pide que el bloque sea del doble del tamaño original. La función
\lstinline!pila_redimensionar! tendrá la siguiente forma:

\begin{codigo-c-plano}
bool pila_redimensionar(pila_t *pila, size_t nuevo_tam)
{
    // No achica la pila menos del valor inicial.
    if (nuevo_tam < MIN_TAM) nuevo_tam = MIN_TAM;

    // Pide la nueva memoria
    void **nuevo = realloc(pila->datos, nuevo_tam * sizeof(void *));
    if (! nuevo) return false;

    // Asigna los nuevos valores
    pila->datos = nuevo;
    pila->tam = nuevo_tam;
    return true;
}
\end{codigo-c-plano}

En este caso se utiliza la función \lstinline!realloc!, que recibe la
dirección actual donde se encuentran los datos, y el nuevo tamaño que se
quiere reservar.  \lstinline!realloc! se encarga de reservar el nuevo bloque,
copiar toda la información que estaba en el bloque viejo al nuevo y liberar el
viejo.

Al igual que en el caso de la creación de la pila, si por algún motivo no es
posible reservar la memoria según se quiere, \lstinline!realloc! devuelve
\lstinline!NULL!.  En este caso, los valores que ya estaban en la pila siguen
estando ahí, simplemente significa que no se ha podido agrandar la porción de
memoria reservada según se había pedido, es por ello que se utiliza un puntero
auxiliar \lstinline!nuevo! y sólo se lo asigna al atributo \lstinline!datos! en
el caso en que la reserva de memoria haya sido exitosa.

\subsection{Destrucción de la pila}

Como ya se dijo, cuando se reserva memoria mediante estas funciones, es
importante luego liberar la memoria reservada, porque de no hacerlo quedan
bloques de memoria inutilizables.  Es por ello que será necesario contar con
una función \lstinline!pila_destruir!, y quien utilice la pila deberá recordar
llamar a esta función al terminar de utilizarla.

\begin{codigo-c-plano}
void pila_destruir(pila_t *pila)
{
    if (pila) free(pila->datos);
    free(pila);
}
\end{codigo-c-plano}

Una vez que se ha llamado a la función \lstinline!free!, la porción de memoria
dinámica deja de estar reservada, ya no es más una porción de memoria válida y
no puede ser accedida por el programa (a menos que se haga una nueva reserva).

\subsection{Disminución del tamaño de la pila}

Finalmente, si bien es posible tener una pila que sólo crezca y nunca se
reduzca, en general es deseable liberar la memoria que no está siendo
utilizada, para que pueda ser usada por otras partes del programa.  De modo
que sería deseable que la pila se reduzca al desapilar, cuando el espacio
ocupado por los elementos en mucho menor que el tamaño de la pila.

\begin{codigo-c-plano}
void* pila_desapilar(pila_t *pila)
{
    if (pila_esta_vacia(pila)) return NULL;

    // Desapila y se guarda el tope.
    pila->largo--;
    void *r = pila->datos[pila->largo];

    // Verifica si hay que achicar, si no puede no hace nada
    if ((pila->tam > MIN_TAM) && (pila->largo <= (pila->tam/4))) {
        pila_redimensionar(pila, pila->tam / 2);
    }

    return r;
}
\end{codigo-c-plano}

En este caso, luego de desapilar el elemento pedido, la función verifica si la
cantidad de elementos ocupa menos de un cuarto del tamaño total de la pila, y
de ser así, reduce el tamaño a la mitad.

\section{Aritmética de punteros}

Las direcciones de memoria en C son valores enteros positivos, el valor
\lstinline!NULL! es equivalente a \lstinline!0!, que es una posición de memoria
inválida.

Los punteros contienen direcciones de memoria asociadas a un tipo en
particular (a excepción de el tipo \lstinline!void!). Como ya se vio, cada
tipo tiene asociado un tamaño en bytes. El compilador de C utiliza esta
información para poder realizar operaciones aritméticas (sumas y restas de
valores enteros) con punteros.

Este es un tema que se presenta en principio complejo, pero que hace que se
pueda operar de forma muy poderosa sobre las porciones de memoria utilizadas.
En particular, es posible utilizar esta técnica para recorrer un arreglo de
valores, sin que necesariamente se los haya declarado como vector. Ejemplo:

\begin{codigo-c-plano}
float fs[MAX_LARGO];
float *pf = fs;           // pf apunta al comienzo del vector fs
pf = pf + 1;              // pf apunta al segundo elemento de fs (&fs[1])
pf = fs + MAX_LARGO - 1;  // pf apunta a el último elemento de fs
\end{codigo-c-plano}

En este caso se declara un vector de valores de tipo \lstinline!float! (que
ocupan 4 bytes cada uno), luego se declara un puntero a valores de tipo
\lstinline!float!, que se inicializa con la posición de memoria del primer
elemento del vector.

Al sumarle 1, sin embargo, la posición de memoria se incrementa 4 bytes, ya
que se trata de un puntero y de esta manera avanza al siguiente
\lstinline!float! del vector.

De la misma manera, en la última línea, se obtiene un puntero a la dirección
de memoria del último elemento del arreglo. \\

Si no se declara el puntero del tipo correcto, en cambio, no es posible operar
de esta manera con las direcciones de memoria.  Es decir que si se hiciera
algo como lo siguiente:

\begin{codigo-c-plano}
void *pv = fs;  // pv apunta al comienzo del vector fs (sin tipo asociado)
pv = pv+1;      // pv apunta al segundo byte de fs *ERROR*
\end{codigo-c-plano}

Se obtendría un puntero a la dirección de memoria del segundo \textbf{byte}
del vector de \lstinline!float!, lo cual podría dar lugar a diversos errores,
ya que si se quiere acceder a la información, se estaría accediendo a 3 bytes
de un valor y 1 byte del siguiente. \\

Es por ello que es posible decir que el operador de acceso a un elemento
\lstinline![]! en C es \textit{azúcar sintáctico}\footnote{Un agregado a la
sintaxis del lenguaje para hacerla más agradable, pero no imprescindible.},
siendo \lstinline!a[10]! sintácticamente equivalente a \lstinline!*(a+10)!,
así como a \lstinline!10[a]!. Claro que este último, si bien válido, hace que
el código sea extremadamente poco legible, por lo que no se lo debe utilizar.

\section{Uso directo de bloques de memoria}

En la biblioteca estándar de C hay varias funciones útiles que acceden
directamente a la memoria, que permiten copiar o inicializar valores, que,
por lo general, serán necesarias cuando se trabaje con bloques de memoria.

Estas funciones asumen que tanto el bloque de memoria origen y destino son
porciones válidas de memoria, y que pueden ser escritas desde el programa.  De
no ser así, el sistema probablemente termine la ejecución del programa al
encontrar un acceso a una porción de memoria inválida, generalmente mediante
el error \textit{Violación de segmento} (\textit{Segmentation fault}).

Además, es importante tener en cuenta que todas estas operaciones tienen un
costo lineal con respecto al tamaño de memoria sobre el cual operan, es decir
que el tiempo requerido para ejecutarlas depende del tamaño de los bloques de
memoria.

\subsection{Copiar contenidos de bloques de memoria (\textit{memcpy()} y
\textit{memmove()})}

Para copiar el contenido de un bloque memoria a otro se puede utilizar la
función \lstinline!memcpy!\footnote{Para más información: \texttt{man 3
memcpy}.}, declarada en el encabezado \lstinline!<string.h>!, cuyo prototipo
es:

\begin{codigo-c-plano}
void *memcpy(void *destino, const void *origen, size_t cantidad);
\end{codigo-c-plano}

La función copia \lstinline!cantidad! bytes desde la posición de memoria
\lstinline!origen! hacia la posición \lstinline!destino! y devuelve un puntero
a \lstinline!destino!. La función asume que origen y destino son bloques que
no se solapan.

Cuando origen y destino se solapan se debe utilizar la función
\lstinline!memmove!\footnote{Para más información: \texttt{man 3 memmove}.},
también declarada en el encabezado \lstinline!<string.h>!, cuyo prototipo es:

\begin{codigo-c-plano}
void *memmove(void *destino, const void *origen, size_t cantidad);
\end{codigo-c-plano}

La función copia \lstinline!cantidad! bytes desde la posición de memoria
\lstinline!origen! hacia la posición \lstinline!destino! y devuelve un puntero
a \lstinline!destino!. De haber solapamiento, se encarga de que no se pierdan
los datos al momento de hacer la copia.

\subsection{Inicialización de un bloque de memoria (\textit{memset()})}

Para inicializar un bloque de memoria con un valor se puede utilizar la
función \lstinline!memset!\footnote{Para más información: \texttt{man 3
memset}.}, definida en el encabezado \lstinline!<string.h>!, cuyo prototipo
es:

\begin{codigo-c-plano}
void *memset(void *direccion, int byte, size_t cantidad);
\end{codigo-c-plano}

Que escribe el valor de \lstinline!byte!, en el bloque de
\lstinline!cantidad! bytes, que empieza en \lstinline!direccion!.  Devuelve un
puntero a \lstinline!direccion!.


%% Copyright (C) Maximiliano Curia <maxy@gnuservers.com.ar>,
%               Margarita Manterola <marga@marga.com.ar>

% Esta obra está licenciada de forma dual, bajo las licencias Creative
% Commons:
%  * Atribución-Compartir Obras Derivadas Igual 2.5 Argentina
%    http://creativecommons.org/licenses/by-sa/2.5/ar/
%  * Atribución-Compartir Obras Derivadas Igual 3.0 Unported
%    http://creativecommons.org/licenses/by-sa/3.0/deed.es_AR.
%
% A su criterio, puede utilizar una u otra licencia, o las dos.
% Para ver una copia de las licencias, puede visitar los sitios
% mencionados, o enviar una carta a Creative Commons,
% 171 Second Street, Suite 300, San Francisco, California, 94105, USA.
\renewcommand{\chaptermark}[1]{\markboth{#1}{}}
\renewcommand{\thesection}{\arabic{section}}
\chapter*{Vectores y punteros}

Una particularidad de C, que puede parecer poco intuitiva al principiante, es
que al utilizar el nombre de una variable de un vector en nuestro código, C
(casi siempre)\footnote{Excepto para sizeof, \& y en el uso en la
inicialización de un vector en la declaración.} usa la dirección de memoria
donde está ubicado el vector.  Es decir, para prácticamente todos los usos, el
nombre de un vector es equivalente a un puntero.

El siguiente código imprime la posición de memoria del vector a:
\begin{codigo-c-plano}
  int a[] = {1, 2, 3, 4, 5};
  printf("%p\n", a);
\end{codigo-c-plano}

Por otro lado,

\begin{codigo-c-plano}
  int *p = a;
\end{codigo-c-plano}

Es una intrucción válida, ya que C guarda la dirección de memoria de a en p.
Sin embargo, esto no quiere decir que un vector sea un puntero.

\begin{figure}[htb]
\centering
\includegraphics{imagenes/vectores-pila}
\caption{Pila de ejecución del código mostrado}
\end{figure}

Además de almacenarse de forma distinta en memoria, a un vector no se le puede
cambiar su posición en memoria, por lo que el siguiente código es inválido:

\begin{codigo-c-plano}
  int b[10];
  b = a; // ESTO NO ANDA
\end{codigo-c-plano}

En una función que recibe un arreglo por parámetro, en realidad C le pasa la
dirección de memoria del vector. Por ejemplo, si definimos la siguiente
función \lstinline!suma!:

\begin{codigo-c-plano}
  long suma(int *datos, int largo)
  {
    long res = 0;
    for(int i = 0; i<largo; i++) {
        res += datos[i];
    }
    return res;
  }
\end{codigo-c-plano}

El puntero \lstinline!datos! tendrá la dirección de memoria del vector recibido, 
\lstinline!datos! es un puntero y no un vector.  Para invocar a la función,
podremos hacerlo de la siguiente forma:

\begin{codigo-c-plano}
    suma(a, 5);
\end{codigo-c-plano}

Siendo \lstinline!a! el mismo vector con el que venimos trabajando. Al
poner el nombre del arreglo, C usa la dirección de memoria de \lstinline!a!
en la invocación a la función \lstinline!suma!.

En el código de \lstinline!suma! vemos que el puntero se usa exactamente
igual que como usaríamos el vector. Nuevamente, C usa la dirección del
arreglo (incluso para el operador \lstinline![i]!), por lo que el uso es
casi idéntico a utilizar un puntero.

En particular, el operador \lstinline!vector[posicion]! es exactamente lo
mismo que escribir \lstinline!*(vector+posicion)! y esto funciona ya que al
sumar un entero a una dirección de memoria el entero se multiplica por el
\lstinline!sizeof! del tipo apuntado (a esto último se lo suele llamar
\textit{aritmética de punteros}).

Para mayor claridad, el tipo de variable recibida (\lstinline!int *datos!)
se puede escribir también como \lstinline!int datos[]!, que es la forma
usual de recibir un vector en C.

\section{Vectores de vectores}

Al declarar cualquier tipo de vector, C requiere que cada posición del
vector tenga exactamente el mismo largo (en bytes), por lo que cada
posición debe ser de un tamaño fijo.

Es por ello que, si se desea armar un vector de vectores (una matriz), se
lo deba hacer de una forma particular.  Por ejemplo, si se desea construir
una matriz de 3 filas y 4 columnas, se lo podría hacer de la siguiente
manera:

\begin{codigo-c-plano}
    int v[][4] = {{1, 2, 3, 4}, {5, 6, 7, 8}, {9, 10, 11, 12}};
\end{codigo-c-plano}

Como se ve, es posible no especificar la cantidad de filas, ya que estas se
especificarán automáticamente al inicializar, pero  es imprescindible
especificar la cantidad de columnas, ya que esto es lo que determina la
medida de cada uno de los elementos del vector \lstinline!v!.

Para acceder a la información del vector, se lo hará de la forma intuitiva:
\lstinline!v[fila][columna]!.

Hasta aquí no resulta demasiado complejo.  Las limitaciones surgen cuando
se quiere poder pasar un vector por parámetro a una función.  La forma
correcta de pasar un vector como el mostrado a una función, sería la
siguiente:

\begin{codigo-c-plano}
void imprimir_matriz(int v[][4], size_t filas);
\end{codigo-c-plano}

Es decir que la cantidad de filas puede ser variable, y se la debe recibir
por parámetro, pero la cantidad de columnas es parte del tipo de la
variable, y se la debe indicar dentro de la declaración de la función.

Esto se debe a que los elementos en memoria se guardan uno a continuación
del otro, y C necesita saber cuántos elementos tiene cada una de las
filas para poder saber cuánto se tiene que desplazar hasta encontrar el
elemento.

\section{Usar un bloque contiguo de memoria como una matriz}

Una forma de poder hacer un manejo genérico de matrices en C es basarse en que
la matriz está guardada en forma contigua en memoria y hacer la cuenta de la
posición que le corresponde a cada elemento.

Siguiendo con el ejemplo anterior, \lstinline!v! como dirección de memoria
es la dirección del elemento \lstinline!v[0][0]!. En el siguiente espacio
dentro del bloque de la memoria, que se encuentra desplazado la cantidad de
bytes que mide un entero estará \lstinline!v[0][1]!. Desplazándose cuatro
veces ese tamaño tenemos \lstinline!v[1][0]!.

Utilizando esta idea, podemos crear una nueva variable \lstinline!x!, que sea
un puntero a enteros que apunta al comienzo de \lstinline!v!. Podemos utilizar
la \emph{aritmética de punteros} vista anteriormente: haciendo
\lstinline!(x + fila*cols)! (donde \lstinline!cols! es la cantidad de
elementos por fila) obtenemos la dirección del comienzo la fila
\lstinline!fila!, esta dirección es el comienzo de un vector de enteros, por
lo que podemos hacer \lstinline!(x + fila*cols)[columna]!, para acceder al
valor de \lstinline!v[fila][columna]!.

La ventaja que obtenemos al hacer esto es que podemos escribir funciones
genéricas, que reciban el tamaño de la matriz por parámetro. Por ejemplo:

\begin{codigo-c}
void imprimir_matriz(void* aux, size_t filas, size_t cols)
{
    int *x = aux;
    for (int i=0; i < filas; i++) {
        for (int j=0; j < cols; j++) {
            printf("%d ", (x + i*cols)[j]);
        }
        printf("\n");
    }
}
\end{codigo-c}

Y la invocación a esta función sería \lstinline!imprimir_matriz(v, 3, 4)!

\section{Vectores de vectores de tamaño variable}

Teniendo en cuenta lo visto anteriormente, si se quiere crear matrices
cuyas columnas sean variables como las filas, que se las pueda pasar por
parámetro a funciones sin importar cuántas columnas tenga, será necesario
recurrir a la memoria dinámica.

Para el caso anterior, por ejemplo, se podría declarar un vector de
punteros, y cada uno de esos vectores inicializarlo como un vector de
enteros:

\begin{codigo-c-plano}
    int *w[filas];
    for (int i=0; i < filas; i++) {
        w[i] = malloc(cols*sizeof(int));
    }
\end{codigo-c-plano}

En este código, \lstinline!w! es un vector de \lstinline!filas! punteros.
Cada uno de estos punteros contiene la dirección de memoria de una porción
de memoria dinámica de tamaño \lstinline!cols*sizeof(int)!.

Al pasar este vector por parámetro a una función, se lo hará de la
siguiente manera:

\begin{codigo-c-plano}
void imprimir_matriz(int** v, size_t filas, size_t cols) {
\end{codigo-c-plano}

Utilizando este formato será posible seguir accediendo a los datos
contenidos en la matriz como \lstinline!v[fila][columna]!.

\begin{figure}[htb]
\centering
\includegraphics{imagenes/vectores-matrices}
\caption{Ubicación de los vectores mostrados en la memoria}
\end{figure}

\section{Cadenas y vectores de cadenas}

Como ya se ha mencionado, las cadenas en C son arreglos de caracteres,
terminados por un caracter \lstinline!\0!.

Si bien es posible declararlas como \lstinline!char cadena[largo]!, lo más
usual es declararlas como \lstinline!char *cadena! y luego asignar una
dirección de memoria apropiada para el dato que se quiera representar.

Cuando se declara una cadena en memoria estática, como en el siguiente
ejemplo:

\begin{codigo-c-plano}
char *cadena = "Algoritmos";
\end{codigo-c-plano}

La variable \lstinline!cadena! contiene una dirección de memoria, en la
cual comienza el arreglo de caracteres, que tiene 11 caracteres (10 letras
y un \lstinline!\0! al final). \\

Si se quiere tener un arreglo que contenga varias cadenas, se deberá operar
de manera similar a la mostrada para las matrices.

\begin{codigo-c-plano}
    char* palabras[] = {"Hola", "que", "tal"};
\end{codigo-c-plano}

Es decir que en este caso la variable \lstinline!palabras! contiene un
arreglo de tres punteros, cada uno de los cuales apunta a una porción de
memoria donde están almacenadas las palabras.


%% Copyright (C) Maximiliano Curia <maxy@gnuservers.com.ar>,
%               Margarita Manterola <marga@marga.com.ar>

% Esta obra est� licenciada de forma dual, bajo las licencias Creative
% Commons:
%  * Atribuci�n-Compartir Obras Derivadas Igual 2.5 Argentina 
%    http://creativecommons.org/licenses/by-sa/2.5/ar/ 
%  * Atribuci�n-Compartir Obras Derivadas Igual 3.0 Unported
%    http://creativecommons.org/licenses/by-sa/3.0/deed.es_AR. 
%
% A su criterio, puede utilizar una u otra licencia, o las dos.
% Para ver una copia de las licencias, puede visitar los sitios
% mencionados, o enviar una carta a Creative Commons, 
% 171 Second Street, Suite 300, San Francisco, California, 94105, USA.

\renewcommand{\chaptermark}[1]{\markboth{#1}{}}
\renewcommand{\thesection}{\arabic{section}}
\chapter*{Ordenamiento recursivo por mezcla, \textit{merge sort}}

El m�todo de ordenamiento \textit{merge sort} es uno de los m�todos de
ordenamiento recursivos, basados en la t�cnica de dividir y conquistar.  Se
lo puede utilizar para ordenar cualquier estructura secuencial (vectores,
listas, etc).

Los pasos de este m�todo de ordenamiento son:
\begin{enumerate}
\item Cuando la longitud del vector sea 0 o 1, se considera que ya se
encuentra ordenado. De no ser as�:
\item Se divide el vector en dos partes de aproximadamente la mitad del
tama�o.
\item Se ordena cada una de esas partes, utilizando este mismo m�todo.
\item Tomando las dos partes ordenadas, se las intercala de forma ordenada,
para obtener el vector original ordenado.
\end{enumerate}

Por ejemplo, si el vector original es \verb+[6, 7, -1, 0, 5, 2, 3, 8]+, se
lo dividir� en dos partes: \verb+[6, 7, -1, 0]+ y \verb+[5, 2, 3, 8]+, que
ser�n ordenadas de forma recursiva.  Luego de ordenarlas, se obtendr�:
\verb+[-1, 0, 6, 7]+ y \verb+[2, 3, 5, 8]+.  Al intercalar ordenadamente
los dos vectores ordenados se obtendr� la soluci�n buscada:
\verb+[-1, 0, 2, 3, 5, 6, 7, 8]+.

\section{Implementaci�n b�sica}

Ser� necesario programar dos funciones.  Por un lado la funci�n
\lstinline!merge_sort!, que ser� la funci�n recursiva encargada de dividir
la lista en dos hasta llegar a la condici�n de corte (cuando la lista tenga
un tama�o menor que 2).

\begin{codigo-c}
void merge_sort(int vector[], int inicio, int fin)
{
    int largo = fin - inicio;
    if (largo < 2) {
        return;
    }

    int medio = inicio + (largo / 2);
    merge_sort (vector, inicio, medio);
    merge_sort (vector, medio, fin);
    merge (vector, inicio, medio, fin);
}
\end{codigo-c}

Se puede ver que esta es una funci�n extremadamente sencilla, cuya �nica
tarea es dividir el vector en dos partes. Por otro lado, ser� necesario
programar la funci�n \lstinline!merge!, que ser� la encargada de intercalar
las partes una vez que est�n ordenadas.

\begin{codigo-c}
void merge (int vector[], int inicio, int medio, int fin)
{
    int pos_1 = inicio;
    int pos_2 = medio;
    int aux[fin-inicio];
    int pos_a = 0;

    // Intercala ordenadamente
    while ( (pos_1 < medio) && (pos_2 < fin) ) {
        if ( vector[pos_1] <= vector[pos_2] ) {
            aux[pos_a] = vector[pos_1];
            pos_a++; pos_1++;
        } else {
            aux[pos_a] = vector[pos_2];
            pos_a++; pos_2++;
        }
    }
    // Copia lo que haya quedado al final del primer vector
    while (pos_1 < medio) {
        aux[pos_a] = vector[pos_1];
        pos_a++; pos_1++;
    }
    // Copia lo que haya quedado al final del segundo vector
    while (pos_2 < fin) {
        aux[pos_a] = vector[pos_2];
        pos_a++; pos_2++;
    }

    // Copia los valores del vector auxiliar al original
    int a = 0;
    int i = inicio;
    while (i < fin) {
        vector[i] = aux[a];
        i++; a++;
    }
}
\end{codigo-c}

Si bien esta funci�n tiene unas cuantas l�neas de c�digo, su tarea no es
muy compleja, simplemente inserta en un vector auxiliar los elementos del
vector que ya se encuentran ordenados, de forma que s�lo se recorren los
elementos una sola vez.

\section{An�lisis de complejidad}

Sea $N$ la longitud del vector. Se pueden hacer las siguientes
observaciones:

\begin{itemize}
\item En la funci�n \lstinline!merge! se ve que el tiempo que se tarda en
intercalar dos vectores de longitud $N/2$ es proporcional a $N$, ya que
todos los elementos se copian una vez al vector auxiliar y luego se los
vuelve a copiar al vector original. Es posible, entonces, utilizar $A * N$
para representar ese tiempo.

\item Si se denomina $T(N)$ al tiempo que tarda el algoritmo en ordenar
un vector de longitud $N$, en la funci�n \lstinline!merge_sort! 
se puede ver que $T(N) = 2 * T(N/2) + A * N$, ya que la funci�n simplemente
se llama a s� misma con dos partes de la mitad de tama�o, y luego a la
funci�n \lstinline!merge! con el tama�o total.

\item Adem�s, tambi�n en \lstinline!merge_sort! se puede ver que el tiempo
necesario para un vector de longitud menor a 2 es s�lo el necesario en
hacer una comparaci�n. Es decir que, $T(1) = T(0) = B$.
\end{itemize}

Estos datos forman una ecuaci�n de recurrencia, para resolverla, se
supondr� que $N = 2^k$, quedando las ecuaciones:

\begin{eqnarray}
T(N) = T(2^k) &=&  2 * T(2^{k-1}) + A * 2^k \\
T(1) &=& B
\end{eqnarray}

Es posible resolver estas ecuaciones utilizando el \textit{m�todo
telesc�pico}.  

\begin{eqnarray}
T(2^k) &=& 2 * T(2^{k-1}) + A * 2^k \\
&=& 2*(2*T(2^{k-2} )+A*2^{k-1} )+A*2^k\\
&=& 2^2*T(2^{k-2} )+ 2*A(2^k)\\
&\vdots&\\
&=& 2^i* T(2^{k-i})+ i * A * 2^i\\
&\vdots&\\
&=&2^k*T(1) + k * A * 2^k\\
&=&2^k*B  + k * A * 2^k
\end{eqnarray}

Como $N = 2^k$ entonces $k=\log_2N$, y por lo que esta resoluci�n demuestra
que $T(N) = B*N+A*N*\log_2N$.

Como $A*N*\log_2N$ es un t�rmino de mayor orden que $B*N$, el orden de este
algoritmo es $O(N*\log_2N)$.  

Los valores de las constantes $A$ y $B$ son importantes a la hora de buscar
la mejor implementaci�n de \textit{merge sort}, pero no para el c�lculo del
orden del algoritmo.

Por otro lado, al analizar el espacio que consume este algoritmo, se puede
ver que para realizar el intercalado, necesita copiar el vector a un
vectora auxiliar, es decir que duplica el espacio consumido.

\section{Implementaciones m�s eficientes}

Si bien el orden del algoritmo \textit{merge sort} ser� siempre
$O(N*\log_2N)$, si se quiere una implementaci�n realmente eficiente del
ordenamiento, ser� necesario hacerle algunas mejoras a la implementaci�n
mostrada.

\subsection{Implementaci�n con un solo pedido de memoria}

El valor $A$ est� asociado al tiempo necesario para ejecutar la funci�n
\lstinline!merge!. Una de las operaciones que se puede eliminar de esta
funci�n es el pedido de memoria para el arreglo auxiliar, 
\lstinline!int aux[fin-inicio];!, ya que esta operaci�n consume tiempo en
pedir la memoria (y luego devolverla, al terminar la funci�n), que podr�a
ahorrarse si se hiciera un �nico pedido de memoria para todo el algoritmo.

Para ello, ser� necesario crear una funci�n adicional, que sea la que haga
el pedido de memoria auxiliar, y luego llame a la funci�n recursiva, con
esa memoria ya reservada.

\begin{codigo-c}
void merge_sort(int vector[], int largo)
{
    int aux[largo]; 
    msort(vector,0,largo,aux);
}
\end{codigo-c}

La funci�n recursiva es ahora \lstinline!msort!, que es pr�cticamente igual
a la vista previamente, siemplemente incluye el pasaje de la variable
auxiliar.

\begin{codigo-c}
void msort(int vector[], int inicio, int fin, int aux[])
{
    int largo = fin - inicio;
    if (largo == 1) {
        aux[inicio] = vector[inicio];
        return;
    }

    int medio = inicio + (largo / 2);
    msort (vector, inicio, medio, aux);
    msort (vector, medio, fin, aux);
    merge (vector, inicio, medio, fin, aux);
}
\end{codigo-c}

Por otro lado, la funci�n \lstinline!merge!, ya no deber� realizar el
pedido de memoria para alojar el vector adicional, sino que trabajar�
directamente sobre la misma porci�n del vector auxiliar que la utilizada
para el vector de valores.

\begin{codigo-c}
void merge (int vector[], int inicio, int medio, int fin, int aux[])
{
    int pos_1 = inicio;
    int pos_2 = medio;
    int pos_a = inicio;

    // Intercala ordenadamente (...)
    // Copia lo que haya quedado al final del primer vector (...)
    // Copia lo que haya quedado al final del segundo vector (...)

    // Copia los valores del vector auxiliar al original
    int i; 
    for (i = inicio; i < fin; i++) {
        vector[i] = aux[i];
    }
}
\end{codigo-c}

De esta forma se logr� evitar tener que estar pidiendo memoria para el
vector auxiliar una y otra vez.  Sin embargo, esto no alcanza para decir
que se cuenta con una versi�n realmente eficiente de \textit{merge sort}.

\subsection{Otras mejoras}

Se puede seguir trabajando sobre el mismo algoritmo para agregarle varias
otras mejoras, como por ejemplo:

\begin{description}

\item[Implementaci�n sin copia in�til de los datos]

Otra operaci�n que consume tiempo in�tilmente es volver a copiar los
datos del vector auxiliar al principal al terminar la funci�n
\lstinline!merge!.

Esta copia puede evitarse si se opera alternadamente
con el vector auxiliar y con el principal, de modo que el vector auxiliar
de una llamada a \lstinline!msort! es el principal de la llamada recursiva
realizada dentro de la funci�n, y as�.

\item[Uso de otros tipos de datos]

En los ejemplos mostrados se han usado vectores de enteros para hacer m�s
simple el ejemplo, pero de la misma forma puede usarse cualquier otro tipo
de dato que tengamos alguna forma de compararlo. O bien hacer una
implementaci�n que no le importe el tipo de dato con el que opera, y use
una funci�n que recibe por par�metro para comparar elementos.

\end{description}




%% Copyright (C) 2010-2013, Maximiliano Curia <maxy@gnuservers.com.ar>,
%               2010-2013, Margarita Manterola <marga@marga.com.ar>

% Esta obra está licenciada de forma dual, bajo las licencias Creative
% Commons:
%  * Atribución-Compartir Obras Derivadas Igual 2.5 Argentina
%    http://creativecommons.org/licenses/by-sa/2.5/ar/
%  * Atribución-Compartir Obras Derivadas Igual 3.0 Unported
%    http://creativecommons.org/licenses/by-sa/3.0/deed.es_AR.
%
% A su criterio, puede utilizar una u otra licencia, o las dos.
% Para ver una copia de las licencias, puede visitar los sitios
% mencionados, o enviar una carta a Creative Commons,
% 171 Second Street, Suite 300, San Francisco, California, 94105, USA.

\renewcommand{\chaptermark}[1]{\markboth{#1}{}}
\renewcommand{\thesection}{\arabic{section}}
\chapter*{Ordenamiento rápido, \textit{quick sort}}

El método de ordenamiento \textit{quick sort} es el más famoso de los
métodos de ordenamiento recursivos, su fama se basa en que puede ser
implementado de forma muy eficiente y en la gran mayoría de los casos tiene
el mismo orden de complejidad que \textit{merge sort}.
Al igual que este último, está basado en la técnica de dividir y conquistar.\\

Los pasos de este método de ordenamiento son:
\begin{enumerate}
\item Cuando la longitud del vector sea 0 o 1, se considera que ya se
encuentra ordenado. De no ser así:
\item Se elige un elemento del vector como \textit{pivote}.  Generalmente
será el primero o el último.
\item Se reordenan los elementos del vector de modo que quede dividido en
tres partes (\textbf{partición}): los elementos menores al pivote, el pivote y los elementos
mayores al pivote. Al terminar este paso, el pivote
queda en su lugar definitivo.
\item Se repite el mismo proceso para cada una de las partes que no
contienen al pivote (los menores y los mayores).
\end{enumerate}

Por ejemplo, si el vector original es \verb+[6, 7, -1, 0, 5, 2, 3, 8]+ y se
elige el primer elemento como pivote (\verb!6!), la partición del vector
será: \verb![-1, 0, 5, 2, 3]!, \verb!6!, \verb![7, 8]!. Se procederá a
ordenar recursivamente \verb![-1, 0, 5, 2, 3]! y \verb![7, 8]!, de modo que
el vector final será \verb![-1, 0, 2, 3, 5, 6, 7, 8]!.

\section{Implementación básica}

En este caso se implementará una función \lstinline!quick_sort!, que se
encargará tanto de realizar la partición, como de llamarse recursivamente
hasta que no haya más elementos que ordenar.

La elección del pivote depende de cada implementación de \textit{quick
sort}, en este caso se elige el primer elemento del vector como pivote.

\begin{codigo-c}
void quick_sort(int vector[], int inicio, int fin)
{
    int pivote = inicio;
    int ult_menor = inicio;

    if ( (fin - inicio) < 2 ) {
        return;
    }

    int i;
    for (i = pivote + 1; i < fin; i++) {
        if ( vector[i] < vector[pivote] ) {
            ult_menor++;
            swap(vector, i, ult_menor);
        }
    }

    // Coloca el pivote al final de los menores y el último
    // menor en el primer lugar.
    swap(vector, pivote, ult_menor);
    // Ordena cada una de las mitades
    quick_sort(vector, inicio, ult_menor);
    quick_sort(vector, ult_menor+1, fin);

}
\end{codigo-c}

El bucle principal de la función recorre los elementos del vector una única
vez, cambiando de lugar aquellos que son menores al pivote para que queden
en la primera parte y que los mayores queden en la segunda.

Una vez terminado este bucle, se coloca el pivote en el medio de ambas
partes, de modo que quede ubicado en su posición final.

La función \lstinline!swap! utilizada en esta porción de código, recibe un
vector y dos posiciones dentro del vector, e intercambia los valores que se
encuentran en esas dos posiciones:

\begin{codigo-c}
void swap(int vector[], int pos_1, int pos_2)
{
    int aux = vector[pos_1];
    vector[pos_1] = vector[pos_2];
    vector[pos_2] = aux;
}
\end{codigo-c}

Esta función puede utilizarse siempre que se necesite intercambiar dos
elementos de un vector.

\section{Análisis de complejidad}

A simple vista, el algoritmo de \textit{quick sort} puede parecer muy
similar al de \textit{merge sort}, ya que en ambos casos se divide a la
lista en dos, y se opera sobre partes cada vez más pequeñas.

Sin embargo, algo importante a tener en cuenta es que en el caso del
\textit{quick sort} el orden que tenga el algoritmo dependerá en una gran
parte de la elección del pivote, ya que no es lo mismo elegir un valor que
se encuentre aproximadamente en el medio, de forma que las dos partes sean
aproximadamente del mismo tamaño, que elegir un valor que se encuentre en
uno de los extremos, de modo que una de las partes mida mucho más que la
otra.

Asumiendo que el valor elegido se encuentra aproximadamente en el medio,
se puede ver que el tiempo requerido para ejecutar el algoritmo es:

\begin{eqnarray}
T(N) &=& A * N + 2*T(N/2) \\
T(1) &=& B
\end{eqnarray}

Donde $B$ es el tiempo requerido por el caso base, y $A$ es el valor asociado
a recorrer el vector y cambiar los elementos de lugar en el bucle
principal.  Puede verse que estas ecuaciones son las mismas que para
\textit{merge sort}. \\

Sin embargo, cuando el pivote elegido no divide ambas partes al medio, el
comportamiento no es tan bueno.  En el peor caso (cuando una parte queda
con todos los elementos menos el pivote y la otra vacía), será:

\begin{eqnarray}
T(N) &=& A * N + T(N-1) \\
T(1) &=& B
\end{eqnarray}

De modo que aplicando el método telescópico, similar al utilizado
anteriormente:

\begin{eqnarray}
T(N) &=& A * N + T(N-1) \\
&=& A * N + A * (N-1) + T(N-2) \\
&=& A * (N + N - 1 + N - 2) + T(N-3) \\
&\vdots&\\
&=&A * (N + N - 1 + N - 2 + \hdots + 2) + B\\
&=&A * \sum_{i=2}^N{i} + B\\
&=&A * \frac{N^2+N}{2} - 1 + B
\end{eqnarray}

Se puede ver que en el peor caso, el orden será $O(N^2)$, mucho peor que el
$O(Nlog_2N)$ visto anteriormente.  Sin embargo, se pueden tomar recaudos
especiales para que este peor caso sea extremadamente improbable, y que en
la práctica se pueda considerar que el algoritmo se comporta como
$O(Nlog_2N)$.

\section{Implementaciones más eficientes}

\subsection{Elección del pivote}

Si se elige el primer elemento (o el último), el algoritmo resulta muy
inconveniente para el caso de una lista que ya se encuentra ordenada, y
este es un caso que en ciertas situaciones es esperable que suceda.

Es por eso que una optimización sencilla es intercambiar el elemento del
medio con el que se vaya a utilizar de pivote antes de comenzar el bucle
principal.

\begin{codigo-c}
    int medio = (fin + inicio) / 2;
    swap(vector, pivote, medio);
\end{codigo-c}

Otras técnicas de elección del pivote incluyen:
\begin{itemize}
\item Elegir un elemento aleatoriamente, esto hace que en promedio sea
mucho más probable tener un buen caso que uno malo, pero no elimina la
posibilidad del peor caso.
\item Recorrer la lista y buscar el elemento que ocupará la posición
central de la lista.  Eso asegura que el orden sea siempre $O(Nlog_2N)$, pero
decrementa mucho la eficiencia del caso base.
\item Elegir tres elementos de la lista (por ejemplo, el primero, el del
medio y el último), y quedarse con el del medio de los tres como pivote.
\end{itemize}

\subsection{Reducción de la cantidad de intercambios}

En la implementación vista, puede suceder que se hagan numerosos
intercambios innecesarios, cuando un elemento ya es menor que el pivote, y
simplemente haría falta avanzar la variable que indica la posición del
último menor.

Una implementación alternativa de {\it quick sort} se basa en esta idea para
tratar de minimizar la cantidad de intercambios.  Se cuenta con dos
variables, que se utilizan para saltear los elementos que no hace falta
cambiar de lugar, y sólo cambiar aquellos que es necesario.

\begin{codigo-c}
void quick_sort(int vector[], int inicio, int fin)
{
    if ( (fin - inicio) < 2 ) {
        return;
    }
    int izq = inicio + 1;
    int der = fin - 1;
    int pivote = inicio;

    // Cambia el del medio con el primero.
    // (optimización para vectores ordenados).
    int medio = (izq + der) / 2;
    swap(vector, pivote, medio);

    while (izq <= der) {
        while ( (izq <= der) && (vector[der] >= vector[pivote]) )
            der--;
        while ( (izq <= der) && (vector[izq] < vector[pivote]) )
            izq++;
        if ( izq < der )
            swap(vector, izq, der);
    }

    // Coloca el pivote al final de los menores y el último
    // menor en el primer lugar.
    swap(vector, pivote, der);
    // Ordena cada una de las mitades
    quick_sort(vector, inicio, der);
    quick_sort(vector, der+1, fin);
}
\end{codigo-c}

Como se puede ver, se ha reemplazado el bucle principal, por otro que
recorre el vector desde ambas puntas hacia el medio, buscando los elementos
que necesitan ser intercambiados.

\subsection{Utilización de otros algoritmos}

En particular para los casos de las partes más pequeñas, al tener menos
elementos, sin importar cuál se elija como pivote, es más probable que se
asemejen al peor caso.

Es por ello que una técnica de optimización puede incluir utilizar un
algoritmo alternativo, como por ejemplo el de ordenamiento por inserción,
para secuencias de pocos elementos. \\

Por otro lado, puede también implementarse un contador que verifique la
profundidad de la recursión y cuando esta exceda el nivel esperado por el
algoritmo, pasar a utilizar otro algoritmo de ordenamiento, como {\it merge
sort} o {\it heap sort}.

\section{Quick sort en la biblioteca estándar de C}

Entre las funciones que provee la biblioteca estándar de C, se incluye una
implementación de quick sort. Dado que la biblioteca estándar está altamente
probada y seguramente contenga optimizaciones avanzadas, es en general una
buena idea usar las funciones que provistas antes que usar las propias.

En el caso del \textit{quick sort}, la función se llama \lstinline!qsort! y se
encuentra definida en el encabezado \lstinline!<stdlib.h>!, su prototipo es:

\begin{codigo-c}
void qsort(void *base, size_t nmemb, size_t size,
           int(*compar)(const void *, const void *));
\end{codigo-c}

Que puede ser intimidante por la cantidad y complejidad de parámetros que
recibe, en gran parte debido a la generalidad del código. \lstinline!base! se
refiere al vector a ordenar, \lstinline!nmemb! es la cantidad de elementos
del vector, \lstinline!size! es el tamaño en bytes de un elemento,
\lstinline!compar! es la función que se debe usar para comparar.

Este tipo de funciones se verán con más detalle más adelante.


%% Copyright (C) 2010-2013, Maximiliano Curia <maxy@gnuservers.com.ar>,
%               2010-2013, Margarita Manterola <marga@marga.com.ar>

% Esta obra está licenciada de forma dual, bajo las licencias Creative
% Commons:
%  * Atribución-Compartir Obras Derivadas Igual 2.5 Argentina
%    http://creativecommons.org/licenses/by-sa/2.5/ar/
%  * Atribución-Compartir Obras Derivadas Igual 3.0 Unported
%    http://creativecommons.org/licenses/by-sa/3.0/deed.es_AR.
%
% A su criterio, puede utilizar una u otra licencia, o las dos.
% Para ver una copia de las licencias, puede visitar los sitios
% mencionados, o enviar una carta a Creative Commons,
% 171 Second Street, Suite 300, San Francisco, California, 94105, USA.

\renewcommand{\chaptermark}[1]{\markboth{#1}{}}
\renewcommand{\thesection}{\arabic{section}}
\chapter*{Compilación de varios archivos en C}

En todo programa es importante modularizar el código de forma que se facilite
la reutilización y se minimice la repetición de código.
En particular, cuando se trata de tipos abstractos de datos, es importante
tener un módulo correspondiente a cada tipo abstracto.

Por otro lado, para que un tipo abstracto de datos sea realmente
\textit{abstracto} es recomendable que la implementación de las operaciones
correspondientes al tipo estén separadas de los prototipos de estas
operaciones, de modo que quien las utiliza se concentre únicamente en cuáles
son las operaciones y no en cómo se llevan a cabo.

\section{Encabezado, implementación y código objeto}

En C, esto se logra mediante la separación de cada módulo en un archivo
\verb!.h!, que contiene las declaraciones de estructuras, constantes y
prototipos de las funciones, que es llamado el \textit{encabezado} del módulo,
y un archivo \verb!.c! que contiene las implementaciones correspondientes.

Cada uno de los archivos \verb!.c! se utiliza para generar un archivo
\verb!.o! que contiene el \textit{código objeto}, es decir el código de
máquina, correspondiente a cada módulo.

Cuando se compila un programa completo, todos los \verb!.o! que se hayan
generado a partir de los módulos programados deben combinarse en un sólo
ejecutable.

\subsection{Inclusión de otros encabezados}

Cuando un módulo requiere de funciones definidas en otros módulos, debe
incluir los encabezados (los archivos \verb!.h!) en los que esas funciones
están definidas.  Esta inclusión se realiza normalmente dentro del encabezado
correspondiente al módulo en cuestión.

Es posible que al momento de construir un programa de tamaño considerable
suceda que hay varios módulos que dependen de otro. De modo que podría suceder
que este otro se incluya varias veces, lo cual es problemático y debe ser
evitado.

Para solucionar este problema, se utilizan las construcciones condicionales
del preprocesador, haciendo que el código definido dentro de un encabezado se
incluya en el programa un única vez. Por ejemplo:

\begin{codigo-c-plano}
#ifndef __ENUM_H
    #define __ENUM_H
    typedef enum {OK, ERROR} estado;
    typedef enum {HUMANO, COMPUTADORA} jugador_t;
#endif
\end{codigo-c-plano}

En este caso, el preprocesador verifica si está definida la variable
\lstinline!__ENUM_H!. De no estar definida, la define y luego define los tipos
enumerados correspondientes a este encabezado.

En cambio, si la variable ya estaba definida, significa que este encabezado ya
fue procesado, con lo cual no se hace nada.


\section{Compilación con \texttt{make}}

Cuando los módulos que componen un programa son muchos, los pasos necesarios
para compilarlo pueden ser muchos y tener que regenerarlos manualmente cada
vez que se los modificque sería una tarea demasiado tediosa.  Es por ello que
existe una herramienta llamada \verb!make!, encargada de realizar todos los
pasos de compilación y necesarios y de hacerlos sólo cuando haga falta.

Esta herramienta utiliza, a modo de configuración, los archivos
\verb!Makefile! en donde se indican los pasos a realizar para la compilación
tanto de los módulos como del programa principal.

En estos archivos, básicamente, se pueden definir variables y reglas para
compilar los distintos módulos.

\subsection{Un \texttt{Makefile} sencillo}

A continuación un ejemplo de cómo puede verse un posible archivo
\verb!Makefile!.

\begin{lstlisting}[language=make, numbers=none]
CFLAGS = -g -Wall -std=c99
EXEC = miprog
OBJ = lista.o pila.o
CC = gcc

all: $(EXEC)

lista.o: lista.c lista.h
	$(CC) $(CFLAGS) -c lista.c

pila.o: pila.c pila.h
	$(CC) $(CFLAGS) -c pila.c

$(EXEC): $(OBJ) miprog.c
	$(CC) $(CFLAGS) $(OBJ) miprog.c -o $(EXEC)
\end{lstlisting}

\subsection{Variables}

En la primera parte se declaran 4 variables, \lstinline!CFLAGS! son los
\textit{flags} (parámetros) de compilación utilizados, en este caso se trata
del parámetro que incluye la información para depuración \lstinline!-g! y el
parámetro para que advierta sobre todos los posibles problemas que el
compilador encuentre, \lstinline!-Wall!.

Luego se declara el nombre que tendrá el programa ejecutable.  En este caso no
tiene extensión, puesto que es un programa para sistemas UNIX (Linux, BSD,
Solaris, etc). Si se estuviera compilando para un sistema Windows, la variable
\lstinline!EXEC! sería \lstinline!miprog.exe!.

Luego se listan cuáles serán los módulos que deberán transformarse en código
objeto, y finalmente se coloca el nombre del compilador.  Separar la
información de esta manera permite que si es necesario hacer un cambio en la
forma de compilar un programa, el trabajo para realizarlo sea mínimo.

Es importante notar que en el \verb!Makefile!, las variables se definen
simplemente con \lstinline!VARIABLE=VALOR!, pero luego para utilizarlas, se lo
hace de la forma \lstinline!$(VARIABLE)!.

\subsection{Reglas de compilación}

Luego de las variables, se definen cada uno de los archivos a generar, los
archivos de los cuales estos dependen, y las acciones a llevar a cabo para
compilar los archivos correspondientes.

Cuando se ejecuta el comando \verb!make! sin ningún parámetro, se ejecuta
automáticamente la primera de todas las reglas, es por eso que esta regla
normalmente se llama \lstinline!all! y simplemente indica cuál es el archivo a
generar.

Luego de esta regla especial, se encuentran las reglas de compilación de cada
uno de los módulos del programa.  Las reglas tienen un formato específico que
se debe cumplir para que funcione el \verb!Makefile!.

\begin{lstlisting}[language=make, numbers=none]
archivo: dependencias
	acciones
\end{lstlisting}

Esto significa que \lstinline!archivo! debe generarse cada vez que
cambie una de las \lstinline!dependencias!, ejecutando las
\lstinline!acciones!.

Es importante notar que para que el \verb!make! funcione correctamente, las
acciones a realizar deben tener un tabulador de separación desde el comienzo
de línea.  Puede haber más de una acción, de a una por línea, siempre que se
mantenga un tabulador de separación. \\

En particular, la regla que se encarga de generar el programa principal es
diferente a las otras, ya que incluye una mayor cantidad de dependencias y de
variables.

\begin{lstlisting}[language=make, numbers=none]
$(EXEC): $(OBJ) miprog.c
	$(CC) $(CFLAGS) $(OBJ) miprog.c -o $(EXEC)
\end{lstlisting}

Esta regla indica que el archivo indicado mediante la variable
\lstinline!$(EXEC)! definida previamente debe generarse cuando cambie cualquiera
de los archivos objeto, ya que si estos cambian, también debe cambiar el
ejecutable final, o bien si cambia el código principal del programa. \\

Es importante notar que mediante estas reglas, \verb!make! no sólo es capaz de
compilar los archivos de forma correcta, sino que también es capaz de
realizarlo sólo cuando sea necesario.  Para ello, verifica que la fecha de
modificación de los archivos listados como dependencias sea anterior al
archivo a generar.  De no ser así, vuelve a generarlo, ya que algo ha
cambiado.

\subsection{Reglas genéricas}

Cuando la gran mayoría de los archivos del programa se compilan de una misma
forma, es posible simplificar el archivo \verb!Makefile!, mediante el uso de reglas
genéricas.

Por ejemplo, para el caso del \verb!Makefile! visto anteriormente, las dos
líneas que generan los archivos \verb!.o! podrían simplificarse en una sola de
la siguiente forma:

\begin{lstlisting}[language=make, numbers=none]
%.o: %.c %.h
	$(CC) $(CFLAGS) -c $<
\end{lstlisting}

Esta regla significa que para generar un archivo \verb!.o! es necesario contar
con un archivo del mismo nombre \verb!.c! y otro del mismo nombre \verb!.h!.
En la regla de compilación se utiliza la variable especial \lstinline!$<!, que
toma el valor de la primera de las dependencias listadas (el archivo
\verb!.c!.).  Existe también \lstinline!$@!, que toma el nombre del archivo
que está siendo generado en esa regla.

\subsection{Acciones comunes}

Además de compilar, es común querer eliminar los archivos generados, de modo
que queden únicamente los archivos fuente del programa.  Esta acción
normalmente se realiza mediante una regla especial llamada \lstinline!clean!.

Si bien puede llevar cualquier nombre, lo más usual es ponerle este nombre ya
que tanta gente la llama así que cualquier programador que se encuentre con un
\verb!Makefile! esperará encontrar una regla con este nombre que realice esta
acción.

\begin{lstlisting}[language=make, numbers=none]
clean:
	rm $(OBJ) $(EXEC)
\end{lstlisting}

Para utilizar esta regla (o cualquier otra que no sea la predeterminada), debe
invocarse el comando \verb!make! con el nombre de la regla como parámetro. Es
decir, \verb!make clean!.

Al igual que con \lstinline!clean!, existen otras acciones comunes que se
suelen incluir en la mayoría de los programas.  Las más conocidas:

\begin{description}
\item[build] Para compilar el código (equivalente a la regla
\lstinline!all! del ejemplo mostrado).
\item[install] Para instalar el código compilado en el sistema.
\item[uninstall] Para desinstalar el programa que haya sido
instalado.
\end{description}

\subsection{Variables comunes}

Así como existen reglas comunes, que la mayoría de los programadores están
acostumbrados a encontrar en los archivos \verb!Makefile!, también existen
variables que suelen estar presentes.

\begin{description}
\item[CFLAGS] Que estaba en el ejemplo mostrado, son los
parámetros pasados al compilador.
\item[LDFLAGS] Son los parámetros pasados al enlazador.
\item[PREFIX] Utilizado cuando hay una regla de instalación, es el
directorio a partir del cual se instalarán los archivos.
\end{description}

\subsection{Reglas, archivos y PHONY}

Cada regla de makefile tiene como finalidad crear un archivo, sin embargo hay
reglas como la regla \verb!clean! que mostramos más arriba que no genera
ningún archivo, tampoco depende de ningún archivo (de hecho borra archivos,
pero esto es parte de la acción y a make no le importa que hace la acción).

Es más, si creamos un archivo llamado \verb!clean! make creerá que la regla
está satisfecha. Para evitar que \verb!make! revise archivos que no vamos a
generar es de bastante recomendable declarar las reglas que no generan
archivos como \verb!.PHONY!, por ejemplo:

\begin{lstlisting}[language=make, numbers=none]
.PHONY: install clean
\end{lstlisting}


% Copyright (C) Maximiliano Curia <maxy@gnuservers.com.ar>,
%               Margarita Manterola <marga@marga.com.ar>

% Esta obra está licenciada de forma dual, bajo las licencias Creative
% Commons:
%  * Atribución-Compartir Obras Derivadas Igual 2.5 Argentina
%    http://creativecommons.org/licenses/by-sa/2.5/ar/
%  * Atribución-Compartir Obras Derivadas Igual 3.0 Unported
%    http://creativecommons.org/licenses/by-sa/3.0/deed.es_AR.
%
% A su criterio, puede utilizar una u otra licencia, o las dos.
% Para ver una copia de las licencias, puede visitar los sitios
% mencionados, o enviar una carta a Creative Commons,
% 171 Second Street, Suite 300, San Francisco, California, 94105, USA.

\renewcommand{\chaptermark}[1]{\markboth{#1}{}}
\renewcommand{\thesection}{\arabic{section}}
\chapter*{Parámetros por línea de comandos}

Se le llaman parámetros de línea de comandos a todo lo que se escriba después
del nombre del programa en la invocación de un comando. Por ejemplo, hemos
dicho que para compilar se usa una línea de la forma:

\begin{verbatim} 
gcc hola.c -o hola
\end{verbatim} 

En este caso, estamos invocando al compilador \verb!gcc! para que compile
\verb!hola.c! y genere el archivo ejecutable \verb!hola!. Esto es posible,
ya que el sistema operativo le pasa al gcc los parámetros que escribió el
usuario al invocarlo, de forma que los parámetros que recibe gcc son:

\begin{codigo-c-plano}
"gcc", "hola.c", "-o", "hola"
\end{codigo-c-plano}

En nuestro código, para recibir estos parámetros se utiliza, tradicionalmente,
la \textit{firma} de la función \lstinline!main! que recibe parámetros:
  
\begin{codigo-c-plano}
int main(int argc, char *argv[]);
\end{codigo-c-plano}

En este caso recibimos en \lstinline!main! los parámetros de la invocación
del programa.  El primer parámetro será la cantidad de parámetros que se
reciben y el segundo parámetro en un vector de punteros a caracteres.

Como se vio anteriormente, cada puntero a caracteres, es la dirección de
memoria de un bloque de caracteres.

\begin{figure}[htb]
\centering
\includegraphics{imagenes/parametros}
\caption{Estructura de los parámetros recibidos por la línea de comandos
al compilar}
\end{figure}

Notar que la primera cadena apuntada por \lstinline!argv!, es el nombre del
comando que se llamó.

\section{Ejemplo de recibir parámetros por línea de comandos}

En UNIX existe un comando llamado \textbf{echo} cuya única función es imprimir todos
los parámetros que se reciben por línea de comandos. Cada parámetro se imprime
con un espacio entre parámetro y parámetro. Este sencillo programa puede
escribirse en C de la siguiente forma.

\begin{codigo-c}
#include <stdio.h>
int main(int argc, char *argv[])
{
    if (argc > 1) {
        printf("%s", argv[1]);
    }
    for (int i=2; i<argc; i++) {
        printf(" %s", argv[i]);
    }
    printf("\n");
    return 0;
}
\end{codigo-c}

En este código imprimimos todos los parámetros recibidos, omitiendo el
nombre del comando (\lstinline!argv[0]!).
Una vez compilado se lo puede invocar de la siguiente forma:
    
\begin{verbatim} 
./echo primer_parámetro  segundo_parámetro   etc
\end{verbatim} 

Y sin importar la cantidad de espacios entre un parámetro y otro el resultado
será: 

\begin{verbatim} 
primer_parámetro segundo_parámetro etc
\end{verbatim}


%% Copyright (C) Maximiliano Curia <maxy@gnuservers.com.ar>,
%               Margarita Manterola <marga@marga.com.ar>

% Esta obra est� licenciada de forma dual, bajo las licencias Creative
% Commons:
%  * Atribuci�n-Compartir Obras Derivadas Igual 2.5 Argentina
%    http://creativecommons.org/licenses/by-sa/2.5/ar/
%  * Atribuci�n-Compartir Obras Derivadas Igual 3.0 Unported
%    http://creativecommons.org/licenses/by-sa/3.0/deed.es_AR.
%
% A su criterio, puede utilizar una u otra licencia, o las dos.
% Para ver una copia de las licencias, puede visitar los sitios
% mencionados, o enviar una carta a Creative Commons,
% 171 Second Street, Suite 300, San Francisco, California, 94105, USA.

\renewcommand{\chaptermark}[1]{\markboth{#1}{}}
\renewcommand{\thesection}{\arabic{section}}
\chapter*{Archivos en C}

En este apunte se ver� una referencia de las funciones y conceptos de archivos
usado en C, resaltando algunas peculiaridades que no se ven en otros
lenguajes. Pero de ninguna manera pretende ser un apunte completo sobre el uso
de archivos en general y se asume cierta experiencia al respecto.

Una de las peculiaridades de C es que, todos los programas al ejecutarse ya
tienen tres archivos abiertos, estos son: la entrada est�ndar
(\textit{stdin}), salida est�ndar (\textit{stdout}) y salida de error
(\textit{stderr}). Los primeros dos son los que usan las funciones de entrada
y salida del usuario, como \lstinline!scanf!  y \lstinline!printf!,
respectivamente. El tercero es un archivo de salida destinado al env�o de
errores de ejecuci�n y por omisi�n saldr�n en la misma salida que los de
salida externa.

Siendo que lo que tiene que ver con archivos es normalmente entrada o salida
del programa, las funciones listadas en este apunte est�n declaradas en el
encabezado \lstinline!<stdio.h>!.

\section{Entrada y salida de una terminal}

La entrada y salida de una terminal en C se comporta de una forma similar a la
lectura y escritura de archivos, por lo que se listan a continuaci�n algunas
de las funciones de entrada y de salida.

Tanto la entrada como la salida est�ndar suelen tener un \textit{buffer}, es
decir una memoria intermedia, en este caso por l�neas, por lo que al intentar
leer de entrada est�ndar mediante \lstinline!scanf!, el programa se quedar�
esperando hasta que se termine una l�nea en la entrada, a�n si s�lo se quiere
leer un caracter. El resto de l�nea no procesada ser� la entrada de las
siguientes llamadas a las funciones de entrada.

Algo similar sucede con la salida por consola, cuando se utiliza
\lstinline!printf!, la salida suele tener tambi�n un \textit{buffer} orientado
a l�neas, por lo que hasta que no se termine una l�nea, la salida no se emite
en la terminal.

\subsection{Manejo de caracteres de a uno}

Sin embargo, no es la �nica opci�n.  Existen otras funciones como:

\begin{codigo-c-plano}
int getchar(void);
\end{codigo-c-plano}

Esta funci�n permite leer un �nico caracter desde la entrada est�ndar,
devuelve el valor del caracter le�do o, en caso de haberse terminado la
entrada, el valor especial \lstinline!EOF!.

De la misma forma, para emitir un �nico caracter por la terminal:

\begin{codigo-c-plano}
int putchar(int c);
\end{codigo-c-plano}

Esta funci�n permite escribir un caracter en la terminal, devuelve el valor
del caracter escrito o bien \lstinline!EOF! en caso de error.

% %%%% scanf y printf ya estaban, pero se podr�a completar mejor.

%\subsection{Entrada y salida con formato}
%
%Como ya se ha visto previamente, para una lectura con formato se utiliza 
%\lstinline!scanf!\footnote{Para m�s informaci�n: \texttt{man 3 scanf}.}, 
%cuyo prototipo es:
%
%\begin{codigo-c-plano}
%int scanf(const char *formato, ...);
%\end{codigo-c-plano}
%
%El primer par�metro es una cadena de formato, que define, entre otras cosas,
%los tipos de las variables a leer. El \lstinline!...!, es una forma de
%declarar una funci�n que recibe una cantidad arbitraria de par�metros. En este
%caso, en esa ubicaci�n van las direcciones de memoria en las que se deben
%escribir los datos leidos. El valor devuelto es la cantidad de valores leidos,
%que puede ser menor a la cantidad pedida, o bien \lstinline!EOF! en caso de
%error.
%
%Para emitir valores por la terminal usamos \lstinline!printf!\footnote{Para
%m�s informaci�n \texttt{man 3 scanf}.}, cuyo prototipo es:
%
%\begin{codigo-c-plano}
%int printf(const char *formato, ...);
%\end{codigo-c-plano}
%
%El primer par�metro es una cadena de formato, que define, entre otras cosas,
%los tipos de las variables a emitir. Los siguientes par�metros ser�n los
%valores a imprimir.

% %%% Cadena de formato

\section{Abrir archivos}

Para abrir un archivo en C se utiliza la funci�n
\lstinline!fopen!\footnote{Para m�s informaci�n: \texttt{man 3 fopen}.}, cuyo
prototipo es:

\begin{codigo-c-plano}
FILE *fopen(const char *ruta, const char *modo);
\end{codigo-c-plano}

El primer par�metro es el nombre del archivo, y el segundo el modo de
apertura, que puede ser:

\begin{tabular}{lp{10cm}}
\verb!r!  & S�lo lectura, se posiciona al principio del archivo. \\
\verb!r+! & Lectura y escritura, se posiciona al principio del archivo. \\
\verb!w!  & Borra el contenido del archivo o crea uno nuevo, s�lo escritura, se
posiciona al principio del archivo. \\
\verb!w+! & Borra el contenido del archivo o crea uno nuevo, lectura y
escritura, se posiciona al principio del archivo. \\
\verb!a!  & Abre para a�adir (escribir al final del archivo). El archivo se
crea si no existe. Se posiciona al final del archivo. \\
\verb!a+! & Abre para leer y a�adir (escribir al final del archivo). El
archivo se crea si no existe. Se posiciona al final del archivo.\\
\end{tabular}

Adem�s, el archivo puede abrirse en modo \emph{archivo de texto} (por omisi�n)
o en modo \emph{archivo binario} (agreg�ndole una \verb!b! al modo). Los
archivos de texto tienen un tratamiento especial para el caracter fin de
l�nea, mientras que con los archivos binarios se accede a los datos en crudo.

El valor devuelto por \lstinline!fopen! es un puntero de tipo
\lstinline!FILE! que representa a los archivos en la biblioteca est�ndar. En
caso de error, el valor devuelto es \lstinline!NULL!.

\section{Cerrar archivos}

Cerrar un archivo es m�s sencillo:

\begin{codigo-c-plano}
int fclose(FILE *archivo);
\end{codigo-c-plano}

Devuelve \lstinline!0! si tuvo exito, o \lstinline!EOF! en caso de error.

\section{Leer o escribir de un archivo}

De la misma manera que \lstinline!getchar!  para leer un caracter de la
entrada est�ndar, existe \lstinline!fgetc!  \footnote{Para m�s informaci�n:
\texttt{man 3 fgetc}.} para leer un �nico caracter de un archivo.

\begin{codigo-c-plano}
int fgetc(FILE *archivo);
\end{codigo-c-plano}

De hecho, la siguiente funci�n es pr�cticamente equivalente a la funci�n
\lstinline!getchar()!.

\begin{codigo-c-plano}
int mi_getchar(void)
{
	return fgetc(stdin);
}
\end{codigo-c-plano}

De la misma forma, existen \lstinline!fputc!, para escribir un caracter a un
archivo, \lstinline!fscanf!, para leer con formato de un archivo,
\lstinline!fprintf!  un archivo.\footnote{Para m�s informaci�n: \texttt{man
3 fputc}, \texttt{man 3 fscanf}, \texttt{man 3 fprintf}.}, para escribir con
formato a Sus prototipos son:

\begin{codigo-c-plano}
int fputc(int c, FILE *archivo);
int fscanf(FILE *archivo, const char *formato, ...);
int fprintf(FILE *archivo, const char *formato, ...);
\end{codigo-c-plano}

Adem�s de estas funciones existen:

\begin{codigo-c-plano}
char *fgets(char *buffer, int tamanio, FILE *archivo);
int fputs(const char *buffer, FILE *archivo);
\end{codigo-c-plano}

La funci�n \lstinline!fgets! lee el archivo hasta encontrar un fin de l�nea,
un fin de archivo o haber llegado a leer \lstinline!tamanio! bytes. Cuando lee
un fin de l�nea lo deja en el \lstinline!buffer!. Devuelve la direcci�n del
\lstinline!buffer! o bien \lstinline!EOF! si se trata de leer estando al final
del archivo.

La funci�n \lstinline!fputs! escribe la cadena apuntada por \lstinline!buffer!
en \lstinline!archivo!. Devuelve la cantidad de bytes escritos o bien
\lstinline!EOF! en caso de error.

Las funciones \lstinline!fgets! y \lstinline!fputs! constituyen la forma
est�ndar de leer o escribir l�neas en un archivo, si bien puede suceder que lo
que se lea no sea una l�nea completa (cuando la l�nea ocupa m�s espacio que
\lstinline!tamanio!.

Si bien tienen un paralelo que trabaja sobre la entrada y salida est�ndar,
esas funciones no se utilizan ya que pueden dar lugar a varios problemas de
seguridad.

\section{Otras funciones de archivos}

Otras funciones que vale la pena mencionar son:

\begin{codigo-c-plano}
int fflush(FILE *archivo);
int feof(FILE *archivo);
\end{codigo-c-plano}

La funci�n \lstinline!fflush! fuerza la escritura de los buffers que est�n
pendientes en el archivo. Devuelve 0 si se ejecut� correctamente, o
\lstinline!EOF! en caso de error. Puede utilizarse para evitar el
comportamiento del buffer por l�neas de las salida est�ndar.

La funci�n \lstinline!feof! devuelve algo distinto de cero si se encuentra al
final del archivo o \lstinline!0! en caso contrario.

\section{Archivos binarios}

Los archivos de texto son sencillos de procesar y faciles de leer a�n fuera
del programa que los usa, el �xito en los �ltimos a�os de los formatos XML,
HTML, SVG, etc, demuestra su gran flexibilidad. Por otro lado, los archivos
binarios permiten almacenar la informaci�n de forma que sea muy eficiente
acceder a ella.

El formato a utilizar en una aplicaci�n se debe decidir seg�n el uso que se le
vaya a dar a los archivos, si se quiere que sean legibles por seres humanos,
si se quiere poder compartir la informaci�n entre aplicaciones, o si
simplemente se quiere poder leer y guardar la informaci�n de la forma m�s
eficiente.

%Es parte del trabajo de un programador decidir el formato de datos a utilizar,
%por suerte para nosotros esto se ve con mucho m�s detalle en otra materia. :)

Las funciones vistas hasta ahora son las m�s utilizadas al trabajar sobre
archivos de texto, estas pueden servir para archivos binarios, pero adem�s se
necesitar�n las siguientes:

\begin{codigo-c-plano}
size_t fread(void *buffer, size_t tamanio, size_t cantidad, FILE *archivo);
size_t fwrite(const void *buffer, size_t tamanio, size_t cantidad, 
				FILE *archivo);
\end{codigo-c-plano}

La funci�n \lstinline!fread! lee \lstinline!cantidad! bloques de bytes de
\lstinline!tamanio! bytes cada uno, de un archivo, almacenandolos en
\lstinline!buffer!. Devuelve la cantidad de elementos le�dos del archivo, en
el caso de estar en el final del archivo devolver� 0.

De la misma forma \lstinline!fwrite! escribe \lstinline!cantidad! bloques de
bytes de \lstinline!tamanio! bytes cada uno en archivo y devuelve la cantidad
de elementos escritos.

\begin{codigo-c-plano}
int fseek(FILE *archivo, long desplazamiento, int origen);
\end{codigo-c-plano}

Se mueve dentro el archivo, \lstinline!desplazamiento! es un valor relativo a
\lstinline!origen!, puede referirse al principio del archivo
(\lstinline!SEEK_SET!), a la posici�n actual (\lstinline!SEEK_CUR!) o al final
del archivo (\lstinline!SEEK_END!). El valor devuelto ser� 0 en caso de exito
o \lstinline!-1! en caso de error.

\begin{codigo-c-plano}
long ftell(FILE *archivo);
\end{codigo-c-plano}

Devuelve la posici�n actual del archivo, o \lstinline!-1! en caso de error.

\section{Ejemplo: Copiar un archivo}

\begin{codigo}{copiar.c}{Copia un archivo}
\begin{codigo-c-plano}
#include <stdio.h>
#include <stdlib.h>

int main(void)
{
	FILE *origen, *destino;
	int  valor;

	origen = fopen("copiar.c","r");
	if ( origen == NULL ) {
		fprintf(stderr, "Error al abrir el archivo origen");
		exit(1);
	}

	destino = fopen("copiar2.c","w");
	if ( destino == NULL ) {
		fprintf(stderr, "Error al abrir el archivo destino");
		exit(1);
	}

	do {
		valor = fgetc(origen);
		if ( valor != EOF ) {
			fputc(valor,destino);
		}
	} while (valor != EOF);
	fclose(origen);
	fclose(destino);

	return 0;
}
\end{codigo-c-plano}
\end{codigo}

En este ejemplo vemos el uso de varias de las funciones mencionadas
anteriormente. El c�digo copia un archivo de un forma muy ineficiente,
leyendolo de 1 caracter. Se muestra tambi�n el uso de \lstinline!stderr!.

Podemos mejorarlo un poco leyendo por lineas en vez de caracter a caracter.

\begin{codigo-c-plano}
enum {MAXLINE = 1024};
...
	char buffer[MAXLINE], *aux;
	do {
		aux = fgets(buffer, MAXLINE, origen);
		if ( aux != NULL ) {
			fputs(buffer, destino);
		}
	} while (aux != NULL);
\end{codigo-c-plano}

Se puede mejorar la eficiencia de este c�digo utilizando \lstinline!fread! y
\lstinline!fwrite!.

% Tomar los nombres de archivo por par�metro 
% Y usar stdin y stdout si no se especifican.
% TODO: Falta, freopen


%\include{apendice}
%\include{referenc}
%
\chapter*{Licencia y Copyright}
\addcontentsline{toc}{chapter}{Licencia y Copyright}

{\noindent
Copyright \copyright\ 2010-2013, Maximiliano Curia <maxy@gnuservers.com.ar> \\
Copyright \copyright\ 2010-2013, Margarita Manterola <marga@marga.com.ar> \\
}

Esta obra está licenciada de forma dual, bajo las licencias Creative Commons: 

\begin{itemize}
\item Atribución-Compartir Obras Derivadas Igual 2.5 Argentina \\ http://creativecommons.org/licenses/by-sa/2.5/ar/
\item Atribución-Compartir Obras Derivadas Igual 3.0 Unported \\ http://creativecommons.org/licenses/by-sa/3.0/deed.es\_AR.
\end{itemize}

A su criterio, puede utilizar una u otra licencia, o las dos. Para ver una copia de las licencias, puede visitar los sitios mencionados, o enviar una carta a Creative Commons, 171 Second Street, Suite 300, San Francisco, California, 94105, USA.

%\typeout{Bibliography}

%\addcontentsline{toc}{chapter}{Referencias}

%\bibliography{referenc}
%\bibliographystyle{alpha}

\end{document}
